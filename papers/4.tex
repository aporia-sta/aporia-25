
\chapter{A Defence of the Interpretational Account of Validity}
\chaptermark{A Defence of the Interpretational Account of Validity}
\chapterauthor{Audrey Hammer,
\textit{University of Cambridge}}

\begin{quote}
Both the interpretational account and the representational account
provide contrasting accounts of validity for natural-language arguments.
While the interpretational account captures formal validity, unlike the
representational account, it does not capture materially valid
arguments. Therefore, materially valid arguments are viewed as
counterexamples to the interpretational account. I motivate why we may
want to defend the interpretational account over the representational
account and then proceed to defend the interpretational account using
the suppressed premise strategy. The first objection to the suppressed
premise strategy is by Stephen Read, who argues that the supressed
premise is redundant. My contribution is to demonstrate how his
objection fails. I also discuss and defend the suppressed premise
strategy against other objections, which concern the nature of the
supressed premise and the problem of modus ponens.
\end{quote}

\vspace{\credgap}

\section*{Introduction}

Validity, a key concept in logic, concerns whether an argument is
truth-preserving. The interpretational account of validity defends the
view that for an argument to be valid it must be formally valid. I turn
first to the importance of logical form, its role in logic, generally,
and validity, specifically. My discussion then moves to the
interpretational account alongside its rival, the representational
account. Both accounts face distinct issues. While I do not hold that
the representational account is incoherent, I do hold that its
formulation has weaknesses that are absent in the interpretational
account, giving a motivation for preferring the latter rather than the
former. Materially valid arguments, which are not formally valid,
present counterexamples to the interpretational account. The remainder
of the essay is devoted to showing how the suppressed premise strategy
can defend the interpretational account against this main objection. The
suppressed premise strategy will in turn be defended against pressing
objections, primarily Stephen Read's objection that the suppressed
premise is redundant. This objection to the supressed premise strategy
aims to prove that there is a contradiction in adding a suppressed
premise to an already materially valid argument, and my contribution is
to show how this objection fails. I then go on to defend the suppressed
premise strategy against a few other objections, including objections
concerning the nature of the suppressed premise and the argument, and
the problem of modus ponens. The result is a defence of the
interpretational account of validity, using the suppressed premise
strategy.

\section*{Understanding the Relation Between Logical Form and Validity}

Logic is considered the science of deduction: it deals with arguments
and their validity. In formal logical languages, like truth functional
logic and first order logic, we can capture validity using the standard
notion of logical consequence. A formal argument is valid if the
conclusion is a logical consequence of the premises. As Owen Griffiths
and Alexander Paseau put it, ``A formal sentence $\phi$ is a logical
consequence of a set of formal sentences $\gamma$ just if every model of $\gamma$ is a
model of $\phi$''.\footnote{Owen Griffiths and Alexander Paseau, \emph{One
  True Logic} (Oxford: Oxford University Press, 2022), 8.} Thus, we can
describe the formal notion of validity for a logical language, using a
model-theoretic notion of logical consequence.

Once we have captured the notion of validity for logical languages, we
can move on to understanding the concept of validity as applied to
natural language, as the accounts of validity that will be discussed are
accounts of validity for natural language. To understand validity as
applied to natural language, we must introduce the concept of logical
form. Logical form is generally considered to be a property of a
sentence of natural language. The logical form of a sentence is when,
keeping the logical constants fixed, the non-logical expressions get
replaced with variables of the appropriate sort. Thus, the logical form
of a sentence can be expressed using a schema. Given this schematic
representation of form, we can follow Alfred Tarski in the view that
logic is topic neutral, because a schema abstracts from the content of
the sentence, only retaining the form of the sentence. For example, take
the following sentence:

\begin{enumerate}[leftmargin=42] 
\def\labelenumi{(\arabic{enumi})}
\item
  Pigeons wear vests and cats wear hats.
\end{enumerate}

\noindent This sentence can be expressed using the following schema:

\begin{enumerate}[leftmargin=42] 
\def\labelenumi{(\arabic{enumi})}
\setcounter{enumi}{1}
\item
  A $\land$ B.
\end{enumerate}

\noindent This is because the logical expression in sentence (1) is ``and'' which
can be formalised using the symbol ``$\land$'', and the non-logical
expressions in the sentence are ``pigeons wear vests'' and ``cats wear
hats'', and thus these expressions are replaced with variables.

One stipulation with this account of logical form, is that it requires
us to have an understanding of what a logical constant is. Thus far
formality has been captured by its topic neutrality, and since a
demarcation of logical notions is crucial to form, it makes sense to
construct this demarcation using this quality of topic neutrality. Here
we can invoke Tarski's account of isomorphism invariance. Tarski defines
logical notions using an analogy from geometry. Just as we may demarcate
particular geometrical objects by their invariance under
transformations, so too can we demarcate logical notions. Thus, ``we
call a notion 'logical' if it is invariant under all possible one-one
transformations of the world onto itself''.\footnote{Alfred Tarski,
  ``What are Logical Notions?,'' \emph{History and Philosophy of Logic}
  7\emph{,} (1986), 149.} To explain this further, we can consider an
isomorphism to be a bijective function, so between two structures there
is a one-one mapping, which preserves all the relevant relations. This
isomorphism is the transformation that Tarski is speaking of. For a
relation to be isomorphic invariant it must remain unchanged over this
sort of transformation. A relation that is isomorphic invariant is thus
indifferent to individual objects. The only notions that do this are
logical notions, and this confirms neutrality. Thus, we can define a
logical notion as being isomorphically invariant and non-logical notions
as not being isomorphically invariant. This allows for the demarcation,
which is necessary to define logical form.

This understanding of logical form can now aid us in capturing the
notion of formal validity for natural language. It is common in the
literature to equate an argument being formally valid with it being
valid in virtue of its form.\footnote{Mark Sainsbury, \emph{Logical
  Forms: An Introduction to Philosophical Logic}, (Oxford: Blackwell,
  2001): 37.} However, using this as a definition for formal validity is
unsatisfactory, for we still need to define being valid in virtue of
form, which I find to be no more informative than formal validity.
Therefore, I define formal validity to be the following: an argument is
formally valid iff it has a form which has only valid instances. An
example of a formally valid argument is:

\begin{enumerate}[leftmargin=42] 
\def\labelenumi{(\arabic{enumi})}
\setcounter{enumi}{2}
\item
  All men are mortal, Socrates is a man $\therefore$ \ Socrates is mortal.
\end{enumerate} 

\noindent The logical form of the argument can be captured using a schema, as
described above. Given the use of quantifiers in (3), the schema of the
argument is simply its first order formalisation (on the obvious
formalisation key):

\begin{enumerate}[leftmargin=42] 

\def\labelenumi{(\arabic{enumi})}
\setcounter{enumi}{3}
\item
  $ \forall x(Fx \rightarrow Gx), \ Fa \ \therefore \ Ga.$
\end{enumerate} 

\noindent There are no invalid arguments with this form, therefore all the
instances of this form are valid, consequently the argument is formally
valid. It is clear from this explanation that this definition of
validity for natural languages coincides with the definition for formal
languages, meaning that a natural-language argument is formally valid
iff its formalisation is valid.

\section*{Two Accounts of Validity}

We can now examine two model-theoretic accounts of validity for
natural-language arguments. Generally, model-theoretic accounts of
logical consequence are now viewed as more successful compared to other
accounts of logical consequence, and the two accounts that are the focus
of this essay are model-theoretic. As such the central thesis of both
accounts understands logical consequence as concerning truth
preservation across models.\footnote{This contrasts with proof-theoretic
  accounts which hold that the nature of logical consequence involves
  there being a proof from the premises to the conclusion.} The first
account is the interpretational account of validity, which originates
from Bolzano but was promulgated by Tarski.\footnote{Jc Beall, Greg
  Restall, and Gil Sagi, ``Logical Consequence'', \emph{The Stanford
  Encyclopaedia of Philosophy} (Summer 2024 Edition); Stephen Read,
  ``Formal and Material Consequences'', \emph{Journal of Philosophical
  Logic} 23, no. 3, (1994): 249.} This account holds that an argument is
valid if there are no possible interpretations of the argument (except
for a reserved class of logical interpretations) where the premises are
true and the conclusion false. An interpretation of an argument is any
argument that has the same logical form as the initial argument. The
second account is the representational account of validity, which holds
that an argument is valid if it is impossible for the premises to be
true and the conclusion false.\footnote{Read, "Formal and Material
  Consequences'', 250.}

The interpretational account only accepts arguments that are formally
valid. The account achieves this by examining different logical
interpretations of the argument; if there is no interpretation that has
true premises and a false conclusion then the argument is considered
valid. On the other hand, the representational account allows for
arguments that are materially valid, alongside those that are formally
valid. Materially valid arguments are arguments in which the validity of
the argument is in part due to the meaning of the non-logical terms
involved. An example of a materially valid argument is:

\begin{enumerate}[leftmargin=42] 
\def\labelenumi{(\arabic{enumi})}
\setcounter{enumi}{4}
\item
  Jill is a paediatrician $\therefore$ \ Jill is a doctor.
\end{enumerate}

\noindent The representational account intends to capture a more ``intuitive''
notion of validity. Defenders of this account hold that materially valid
arguments are contained within this intuitive notion of validity, and so
an account of validity must capture material as well as formal validity.
This belief is rooted in the idea that there is an analytic connection
between certain words or phrases, and these connections make the
argument valid, even though the argument is not formally valid.

The main objection to the interpretational account is that it is subject
to counterexamples, which take the form of materially but not formally
valid arguments. To establish the success of the interpretational
account we must meet this objection. One example of a materially but not
formally valid argument is (5) above, and another is:

\begin{enumerate}[leftmargin=42] 
\def\labelenumi{(\arabic{enumi})}
\setcounter{enumi}{5}
\item
  Adam is taller than Bill and Bill is taller than Cathy $\therefore$ \ Adam is
taller than Cathy. 
\end{enumerate}

\noindent Neither of these arguments is formally valid, since there are invalid
arguments with the same form as (5) and (6). The interpretational
account would not accept that they are valid arguments given there are
interpretations of (5) and (6) for which the premises are true and the
conclusion false. A formalisation of these arguments in first order
logic reveals their logical form:

\begin{enumerate}[leftmargin=42] 
\def\labelenumi{(\arabic{enumi})}
\setcounter{enumi}{6}
\item
  $Fa \ \therefore \ Ga$ 
\item 
  $ (Tab \land Tbc) \ \therefore \ Tac $
\end{enumerate}

\noindent Another interpretation of each of these arguments demonstrates the point further:

\begin{enumerate}[leftmargin=42] 
\def\labelenumi{(\arabic{enumi})}
\setcounter{enumi}{8}
\item
  Pat is a postman \ $\therefore$ \ Pat is a father.
\item 
  Alice is friends with Bonnie and Bonnie is friends with Carl \ $\therefore$ \ Alice is friends with Carl.
\end{enumerate}

\noindent These arguments are clearly invalid, yet they have the same logical form
as (5) and (6), respectively. It is due to these alternative
interpretations that (5) and (6) are not valid.

However, the arguments (5) and (6) would be accepted under the
representational account due to this account's use of modality. The
representational account identifies logical consequence with
metaphysical consequence. The reference to ``impossible'' in the
representational account is a modal notion, whereas the interpretational
account does not include such modal notions. The reference to ``no
possible interpretations'' in the interpretational account may be made
actual using substitutional classes, and thus does not need to rely on
an analysis of modality.\footnote{Read, "Formal and Material
  Consequences'', 252.} Yet, it is because of its use of modality that
the representational account can attribute validity to (5) and (6), for
there is no possible world where the premises of (5) and (6) are true
and the conclusion false.

On the other hand, modality is an issue for the representational
account, for it requires that we have an analysis of
modality.\footnote{It should be noted that this conversation concerns
  analyses of the metaphysical notion of modality, which is distinct
  from a discussion of modal logic, which is considered to be well
  understood. Metaphysical modality deals with the fundamental nature of
  modal notions, whereas modal logic is a formal system which reasons
  about sentences containing modal operators.} Commonly, modality is
cashed out in turns of possible worlds. This prompts the question of
what a possible world is. The answers to this question are
controversial. We have modal realists, like David Lewis, who endorse a
view that possible worlds exist, as real concrete entities.\footnote{David
  Lewis, \emph{On the Plurality of Worlds}, (Basil Blackwell, 1986) 2-3,
  86.} Adopting this analysis for our account of validity would also
mean adopting the ontological commitments of this account. Other
analyses of modality include modal sceptics, who deny that modal
statements can be known. In adopting this approach, we could not know
whether our arguments are valid, which is entirely counterintuitive.
While there are some more modest approaches to modality, like those
taken by Stalnaker\footnote{Robert C. Stalnaker, ``Possible Worlds,''
  \emph{Noûs} 10, no. 1, (1976): 65-75.} and Adams\footnote{Robert
  Merrihew Adams, ``Theories of Actuality,'' \emph{Noûs} 8, no. 3,
  (1974): 211-231.}, there are still issues surrounding whether these
accounts can provide a reductive analysis. This is all to say that while
modality is often invoked in philosophical topics, the debate
surrounding modal notions is not uncontroversial, and thus any time it
is invoked in a theory, that theory faces the same controversies. This
is not to say that modal notions should never be used in philosophical
theories, but just that we should be aware of the commitment and, all
things being equal, adopt theories without modal notions. This gives us
a motivation to prefer the interpretational account over the
representational account. Indeed, Read, who accepts the representational
account over the interpretational account, admits that the lack of modal
notions in interpretational account is a possible motivation to prefer
this account rather than the representational account.\footnote{Read,
  "Formal and Material Consequences'', 252.}

While this general criticism concerning the use of modal notions is
important to note, there is a more specific problem with the
representational account; namely, the identification of logical
consequence with metaphysical consequence then provides no account of
the importance of formality in logical consequence.\footnote{Beall,
  Restall, and Sagi, ``Logical Consequence.''} Similarly, the account
does not provide a basis for distinguishing between logical and
non-logical vocabulary. This is because the representational account
determines that all expressions used in the argument contribute to the
validity of the argument. Consequently, the representational account
undermines the topic neutrality of logic.

Given that the representational account faces the above challenges, I
suggest that this should motivate us to adopt the interpretational
account instead. While I do not view these issues as being
insurmountable, I simply hold that if there is an alternative we should
favour it. If the problem of counterexamples to the interpretational
account can be overcome, then this account becomes a preferrable
alternative to the representational account of validity. I devote the
remainder of this essay to considering and defending a possible solution
the interpretational account can adopt to resolve the problem of
counterexamples. This solution is the suppressed premise strategy.

\section*{The Suppressed Premise Strategy}

The suppressed premise strategy (hereafter SPS) can be employed by the
interpretational account to overcome the problem of materially valid
arguments. SPS holds that materially valid arguments have suppressed
premises which when revealed make the argument formally valid, and thus
valid under the interpretational account. These suppressed premises are
true given they usually explicitly reveal true analytic connections
between words.\footnote{Read views these suppressed premises not just as
  true but as logically true because he associates logical truth with
  analytic truth (Read, ``Formal and Material Consequences'', 258).
  Since I have not made this association, I will avoid understanding
  suppressed premises as logically true.} Since they are true, the
addition of the suppressed premise is largely unproblematic, although
this claim will be defended further.

SPS applied to the argument (5) gives:

\begin{enumerate}[leftmargin=42] 
\def\labelenumi{(\arabic{enumi})}
\setcounter{enumi}{10}
\item
  Jill is a paediatrician, all paediatricians are doctors $\therefore$ \ Jill is a doctor.
\end{enumerate}

\noindent This argument can be formalised as follows:

\begin{enumerate}[leftmargin=42] 
\def\labelenumi{(\arabic{enumi})}
\setcounter{enumi}{11}
\item
  $Fa, \ \forall x(Fx \rightarrow Gx) \ \therefore \ Ga$
\end{enumerate}


\noindent There are no possible interpretations of the argument (11) that will
have true premises and a false conclusion, thus under the
interpretational account (11) is valid, although (5) remains invalid. Of
course, this strategy applies to (6), where the suppressed premise is
that ``taller than'' is transitive. No suppressed premise can be added
to (9) or (10), since it is not true that all postmen are fathers, there
is no analytic connection between being a postman and being a father,
and the relation ``being friends with'' is not transitive.

\subsection*{The Redundancy Objection}

The first objection to SPS is put forward by Read and states that the
suppressed premise is either false or redundant, and since it cannot be
false it must be redundant. \footnote{Read, ``Formal and Material
  Consequences," 257-9.} Read gives his argument as follows:

\begin{quote}
The extra premise is strictly redundant. For if the original argument
were invalid, the added premise would not be logically true. Given that
it is logically true, it follows that the unexpanded argument was
already valid. Hence it was (logically) unnecessary to add the extra
premise.\footnote{Read, "Formal and Material Consequences", 259.}
\end{quote}

\noindent This objection is best demonstrated using an example. Take argument (5),
which is considered invalid under the interpretational account. Read
says that because of its invalidity, it is possible for the premises of
(5) to be true and the conclusion of (5) to be false. This entails that
it is possible for Jill to be a paediatrician but not be a doctor. Yet
the suppressed premise for this argument is that ``all paediatricians
are doctors'', clearly contradicts the possibility Jill is a
paediatrician and not a doctor. It follows if we accept that (5) is
invalid, then we also accept that the suppressed premise is false. Yet
this suppressed premise is true, so the initial assumption that (5) is
invalid must be false, and therefore the addition of the suppressed
premise is made redundant for it is not necessary for the argument to be
considered valid. According to Read, the suppressed premise's redundancy
means we should reject the interpretational account in favour of the
representational account.

Read's objection, while presented convincingly, lacks any actual force.
This is due to a key error it makes: it presupposes the representational
account, when it should presuppose the interpretational account. It is
not the case that (5) is invalid because the premise ``Jill is a
paediatrician'' is compatible with it being false that ``Jill is a
doctor'', which (if true) is what the representational account would
suppose, rather (5) is invalid because there is an interpretation of (5)
for which the truth of the premises is compatible with the falsity of
the conclusion. (9) is an interpretation of (5) for which it is
compatible that it is true that ``Pat is a postman'' and false that
``Pat is a father'', and therefore (5) is considered invalid under the
interpretational account. Under the interpretational account, nothing
specifically is said about the premises of (5), and so Read is wrong to
infer that attributing invalidity to (5) will make the suppressed
premise false. Since Read is wrong to assert that the invalidity of the
argument shows the suppressed premise's falsity, he cannot then infer
that since the suppressed premise is true, it must therefore be
redundant. Under the representational account, invalidity is saying
something about the specific premises of the argument under
consideration. Yet under the representational account a materially valid
argument, like (5), would not be considered invalid.

Some may reply here that I am begging the question: why is it that we
should assume the interpretational account and not the representational
account? However, this line of thought is also mistaken. Read clearly
starts by assuming that materially valid arguments are invalid, which is
only the case under the interpretational account, not the
representational account. From this assumption of invalidity, he
attempts to prove a contradiction, but then uses the representational
account's understanding of validity in this contradiction, even though
the representational account would not attribute invalidity to something
that is materially valid. However, if the interpretational account is
used, then there is no contradiction in using SPS. In addition, this
strategy is only used by the interpretational account. Thus, Read must
assume the interpretational account if he is going to show a
contradiction; given he does not use the interpretational account in his
objection and that even if he did use the interpretational account there
would be no contradiction, this implies that his objection holds no
weight.

\subsection*{Objections about the Nature of the Suppressed Premise and the
Argument}

A second problem for SPS is that we have not been committed to the view
that the suppressed premise is logically true. This may lead to the
question: why is it acceptable to add to an argument an extra premise
that is not logically true? Surely only logically true propositions may
be added to the premises of an argument to retain the same argument. To
answer this question, an important point must be reiterated: I do not
agree that the argument prior to the addition of the suppressed premise
is the same argument as the argument after the addition of the
suppressed premise. To me this point is obvious, for the two arguments
have different properties: one argument is valid, the other invalid, and
they have a different number of premises. Since we are speaking of two
different arguments, I do not need to prove that the first argument is
``retained'' in the second. However, this does not mean SPS can be used
on any argument. If the premise ``all postmen are fathers'' is added to
(9) then we have a new argument:

\begin{enumerate}[leftmargin=42] 
\def\labelenumi{(\arabic{enumi})}
\setcounter{enumi}{12}
\item
  Pat is a postman, all postmen are fathers $\therefore$ \ Pat is a father.
\end{enumerate}

\noindent (13) is a valid argument, but we should not consider (13) to be using SPS. Therefore, we must identify what differentiates (11) from (13), and
why (11) is determined as using SPS and thereby linking it closely with
(5) in a way that (13) is not linked with (9). The difference is that
the suppressed premise revealed in (11) that ``all paediatricians are
doctors'' is true, but the premise ``all postmen are fathers'' is not
true. Indeed ``all paediatricians are doctors'' is an analytic truth.
However, it is not necessary that this be considered a logical truth. To
begin with, there seems to be no necessity to consider analytic truths
to be logical truths, particularly if we retain the commonly held view
that logic has no special content. And secondly, the goodness of an
argument can be characterised by whether it is sound, i.e., it is valid
and has true premises, which does not require the premises to be
logically true. So long as the suppressed premise is true, its addition
to the argument does not hinder the chances of the argument being sound
and should in fact improve this since the argument will now be formally
valid. Since one of the characteristics of a suppressed premise is that
it is true, there is no issue that it is not logically true. Considering
(13), the premise ``all postmen are fathers'' cannot be a suppressed
premise of the argument (9) for it is not true. Therefore, the
suppressed premise does not need to be logically true, but this does not
mean that SPS can be applied to any argument to make it valid.

Moreover, we may consider that SPS might even allow us to consider
contingent truths as suppressed premises. Let us suppose that it were a
contingent fact that ``all postmen are fathers'', then it might make
sense to consider this to be a suppressed premise of argument (9). Say
Mr. Black presented argument (9) to Mr. White and both Mr. Black and Mr.
White were aware that ``all postmen were fathers'', then the argument
might be accepted as sound in the rhetoric (even though it is not
formally valid) because both understood that the argument has a
suppressed premise, and that Mr. Black in fact meant to make the
argument (13). Now suppose Mr. Smith questioned the validity of the
argument because he was not aware that it was a contingent fact that
``all postmen were fathers''. Yet, once this would be revealed to him,
Mr. Smith would certainly accept the validity of the argument.
Therefore, we may accept that a suppressed premise may be contingently
true, and it becomes clear that only truth, and not logical truth, is
necessary for the suppressed premise.

A counterexample to this argument has been pointed out to me.\footnote{By
  Owen Griffiths, in personal communication.} This is that if we take
the argument:

\begin{enumerate}[leftmargin=42] 
\def\labelenumi{(\arabic{enumi})}
\setcounter{enumi}{13}
\item
  I am a philosophy student $\therefore$ \ puppies are cute.
\end{enumerate}

\noindent This is clearly invalid. But if the conditional ``If I am a philosophy
student then puppies are cute'' is added as a suppressed premise to
(14), then we get the new valid argument:

\begin{enumerate}[leftmargin=42] 
\def\labelenumi{(\arabic{enumi})}
\setcounter{enumi}{14}
\item
  If I am a philosophy student then puppies are cute, I am a
philosophy student $\therefore$ \ puppies are cute.
\end{enumerate}

\noindent It appears there is no problem with adding this conditional if we take
the view that suppressed premises only need to be contingently true, and
not analytically true, because considered as a material conditional it
is true (the antecedent and consequent are true). This seems to be a
problem for the strategy, as it might allow for many arguments like
(14), that have true premises and true conclusions yet are not formally
or materially valid, to be valid by adding these conditionals as
suppressed premises.

My response to this argument is to say that these conditionals are
indicative conditionals, not material conditionals, which means they
involve a different treatment. An indicative conditional is the
conditional of natural language, and the current discussion is about the
validity of natural-language arguments, so it makes sense to speak of
indicative conditionals rather than material conditionals. We may then
consider views of indicative conditionals which hold that their truth
values are different to those of material conditionals, and as such we
can formulate a view that holds that ``If I am a philosophy student then
puppies are cute'' is false. For instance, we might hold that an
indicative conditional is true iff it is assertable and is in turn
assertable iff it passes the Ramsey test. The Ramsey test is a test for
the assertability of a conditional, it holds that a conditional is
assertable if someone were to add the antecedent to her set of
suppositions, she would also have to add the consequent. ``If I am a
philosophy student then puppies are cute'' would clearly fail the Ramsey
test. Thus, we can still consider that the suppressed premise may be
true without the above presenting as a counterexample.

I have only given a rough sketch of a possible response to the objection
suggested above, and while there are many problems with associating the
truth conditions of an indicative conditional with those of the material
conditional, there are still some who adopt this view. However, the
conditional suggested is one where the antecedent and the consequent are
both true and yet have nothing to do with each other. This sort of
conditional is itself a problem case for someone who holds this
truth-functional view of the indicative conditional, suggesting that
there is something wrong with equating the indicative conditional with
the material conditional. However, if the reader insists on the
indicative conditional and the material conditional having the same
truth value, even in cases where the antecedent and consequent have no
relation to each other, then this reader may simply choose to reject
this section on contingent truth and hold that the suppressed premise
must be an analytic truth. This does not detract from the fact that the
suppressed premise is not a logical truth. Of course, the reader may
still object to the idea of analytic truth. However, this paper defends
the interpretational account against the counterexample of material
valid arguments, which themselves rely heavily on the notion of
analyticity. So, if the reader places no importance on the analytic
connections between words, then there is no forceful objection to the
interpretational account and no need for SPS to begin with.

A third objection connects to my answer to the second objection. I have
stated that the two arguments, the argument prior to the addition of the
suppressed premise and the argument after this addition, are two
different arguments. This may lead one to ask, ``what connects the two
arguments?'' The answer to this is simple: they both have the same aim.
The aim of an argument is an imprecise and informal notion; however, I
want to use it to capture an intuitive idea. The two arguments share the
same conclusion, and their aim is to use true (and very similar)
premises to arrive at this conclusion. Suppose that Jones is having a
discussion of Jill's profession; he would be just as happy receiving the
argument (11) as he would be receiving the argument (5), possibly even
happier receiving (11) if he is unaware that a paediatrician is a kind
of doctor (or if he is a logician who has a strong appreciation for
formal validity). However, Jones would be disappointed if instead of
receiving either of these arguments he received (3), for instance, which
clearly has nothing to do with Jill or her profession. The aim of the
arguments is informal, and the setting for which Jones might accept or
reject them, as described, is also informal. The arguments are connected
by this informality. The matter of validity in logic is strictly a
formal matter, and thus there is a distinct difference between (5) and
(11).

\subsection*{The Problem of Modus Ponens}

The final problem I shall explore in relation to SPS is the problem of
modus ponens. A modus ponens is a deductive argument of the following
form:

\begin{enumerate}[leftmargin=42] 
\def\labelenumi{(\arabic{enumi})}
\setcounter{enumi}{15}
\item
  $A, \ A \rightarrow B \ \therefore \ B$
\end{enumerate}

\noindent Modus ponens is discussed by both Read and Timothy Smiley, in very
different ways.\footnote{Read, "Formal and Material Consequences,"
  259-62; Timothy Smiley, "A Tale of Two Tortoises", \emph{Mind} 104,
  no. 496, (1995): 727.} They both view modus ponens as having a similar
form to SPS but speak of different consequences related to this
similarity. Below, I address both in turn.

The problem that Read notes with modus ponens is that the major premise
of this argument (16) is either false or redundant. While his discussion
of this problem is limited, he links it with SPS by arguing that in both
cases the additional premise ``adds psychological perspicuity
{[}\ldots{]} But at the same time, it is not essential''.\footnote{Read,
  "Formal and Material Consequences," 262.} To some extent I disagree
with both points. Considering the second point, the suppressed premise
and the major premise in the modus ponens argument are vital in making
the argument valid, and thus are essential to the argument. On the first
point, there is some sense in which adding the suppressed premise and
the major modus ponens premise do add psychological perspicuity, but it
does not necessarily always do this or do this to the extent Read may be
suggesting. In cases where both parties implicitly know the suppressed
premise, its addition to the argument may not provide any psychological
clarity, only logical infallibility. This idea is strengthened when
considering that most of the arguments we make in everyday life have
suppressed premises and we do not seem to need to reveal these
suppressed premises for psychological reasons.\footnote{Smiley, "A Tale
  of Two Tortoises," 727.} Rather we tend to reveal suppressed premises
for logical reasons. Given we are holding this discussion in the domain
of logic, we may accept the resemblance between SPS and modus ponens
while still rejecting Read's assertion of redundancy.

Smiley's discussion of this matter refers to a paradox that seems to be
presented by modus ponens and the addition of the suppressed premises.
The paradox in question originated from Lewis Carroll, who wrote:

\begin{quote}
If I grant (A) All men are mortal, and (B) Socrates is a man, but not
(C) The sequence "If all men are mortal, and if Socrates is a man, then
Socrates is mortal" is valid, then I do not grant (Z) Socrates is
mortal. Again, if I grant C, but not A and B, I still fail to grant Z.
Hence, before granting Z, I must grant A and B and C. {[}Now consider{]}
(D) If A and B and C be true, then Z is true.\footnote{Charles Lutwidge
  Dodgson, \emph{Lewis Carroll's Symbolic Logic}, W. W. Bartley III,
  ed., (Clarkson Potter, 1977), 472.}
\end{quote}

\noindent This becomes paradoxical when we observe an infinite regress occurring
where we must grant (A), (B), (C), (D), and a further (E) If A and B and
C and D be true, then Z is true, yet we can think of an infinite number
of propositions that must be granted before it seems that Z is granted.
We can view (C), (D), etc, as suppressed premises of the argument that
Carroll reveals in his paradox. This leads Smiley to comment that
``Lewis Carroll was doomed to detect suppressed hypothetical premises
even in logically valid arguments, and incidentally to force them all
into the straitjacket of modus ponens''.\footnote{Smiley, "A Tale of Two
  Tortoises," 727.} If these are considered to be suppressed premises
then there is a problem for SPS, for these can be added to any argument,
and make the argument paradoxical. In addition, this does not seem to be
what the strategy intends. To solve this, we can examine the
characteristics of the suppressed premise, which is that its addition
will make the argument formally valid. Yet the arguments that Lewis
Carroll imagines are already valid arguments, thus SPS should not be
employed in these cases. Smiley's examination of the problem also points
out that the specific wording of the paradox is crucial for its
paradoxical nature but is itself flawed. Lewis Carroll ``lacked any
distinct conception of a deduction as opposed to the assertion'', and it
is this confusion that leads to paradox. \footnote{Smiley, "A Tale of
  Two Tortoises," 727.} By this Smiley means that (C) is not an
assertion but a deduction, and so it must be granted, but Carroll seems
to think that it is merely an assertion that can be accepted or denied.
Hence, this paradox does not show that even valid arguments might have
suppressed premises that lead to paradox, thus this objection presents
no issue to the use of SPS.

The characterisation I have given of SPS prevents contradiction and I
have shown how it is able to overcome objections about the redundancy of
the suppressed premise, as well as more generally the nature of the
suppressed premise and the nature of the arguments to which it pertains.
Finally, I discussed the problem of Modus Ponens, showing two ways it
relates to SPS, and that this does not impact the use of the strategy.
Thus, SPS is a viable addition to the interpretational account and
explains the relation of material validity to validity, without a need
to adopt the representational account. Hence by defence of the
interpretational account succeeds and preferred to the representational
account.

\refsection

\begin{hangparas}{\hangingindent}{1}
Adams, Robert Merrihew. ``Theories of Actuality.'' \emph{Noûs} 8, no. 3
(1974), 211-231.

Beall, Jc, Greg Restall, and Gil Sagi "Logical Consequence",~\emph{The
Stanford Encyclopedia of Philosophy~}(Summer 2024 Edition), Edward N.
Zalta \& Uri Nodelman~(eds.),
\newline
\url{https://plato.stanford.edu/archives/sum2024/entries/logical-consequence}

Dodgson, Charles Lutwidge \emph{Lewis Carroll's Symbolic Logic}. W. W.
Bartley III, ed. Clarkson Potter, 1997.

Griffiths, Owen, and Alexander Paseau. 2022. \emph{On True Logic: A
Monist Manifesto.} Oxford University Press, 2022.

Lewis, David. \emph{On the Plurality of Worlds.} Basil Blackwell, 1986.

Read, Stephen. ``Formal and Material Consequences.'' \emph{Journal of Philosophical Logic} 23,
no. 3 (1994): 247-265.

Sainsbury, Mark. \emph{Logical Forms: An
Introduction to Philosophical Logic.} Blackwell, 2001.

Smiley, Timothy. ``A Tale of Two Tortoises.'' \emph{Mind} 104, no. 416 (1995):
725-736.

Stalnaker, Robert C. "Possible Worlds." \emph{Noûs} 10, no. 1, (1976): 65-75.

Tarski, Alfred. ``What are Logical Notions?''
\emph{History and Philosophy of Logic} 7, 1986): 143-154.
\end{hangparas}
