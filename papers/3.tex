\chapter{Defending Williamson's Explanatory Challenge to Contingentism}
\chaptermark{Defending Williamson's Explanatory Challenge to Contingentism}
\chapterauthor{Koda Li,
\textit{Brown University}}

\section{Introduction}
In his book \emph{Modal Logic as Metaphysics}, Timothy Williamson developed a series of arguments against contingentism and in favor of necessitism. I outline the two theses in the following: 
\begin{quote}
(Contingentism) \hspace{\labelsep} $\Diamond \exists x \Diamond \neg \exists y x=y$ \smallskip \\
Informally, some things could have not existed. \smallskip \\
``The table could have been destroyed in the making process and therefore does not exist." \bigskip \\
(Necessitism) \hspace{\labelsep} $\Box \forall x \Box \exists y x=y$.  \smallskip \\
Informally, everything necessarily exists. \smallskip \\
``This table, the person John, and all other things exist necessarily."
\end{quote}
Williamson's arguments are complex and intricate. This paper will focus on one particular challenge he raised to contingentism in Chapter 6 of his book and various responses toward this challenge. The paper is structured as the following: Section 2 reconstructs Williamson's challenge; Section 3 explains two ``trivialization" worries about this challenge and respond to them on Williamson's behalf; Section 4 develops a substantive response to Williamson's challenge and criticize it on Williamson's behalf. I argue that Williamson's challenge is successful and contingentists have considerable dialectical disadvantages.  

\section{Williamson's Explanation Challenge to contingentism}
Williamson raised a challenge to contingentists who accept (Comp_M) in high-order modal logic.\footnote{The background logic Williamson assumes is the one developed in his Chapter 5, p. 225. What is of significance is that the underlying modal logic is S5. This paper will not tap into the debate of which modal logic is the correct modal logic. I will assume Williamson's logic and develop challenges and responses.}
Below is (Comp_M): 
\begin{quote}
(Comp_M) \hspace{\labelsep} $\vdash \exists X \Box \forall x (Xx \ba A)$ \smallskip \\ where $A$ is a metalinguistic variable ranging over formulas. 
\end{quote}
Informally, (Comp_M) says that for any formula $A$, there is some property that something instantiates just in case $A$ is true.

I will first say something to motivate (Comp_M) before getting into Williamson's challenge using this principle. For one, (Comp_M) is a very attractive higher-order logic principle, for it says roughly that given any formula A, one can define a property such that necessarily, something has it just in case A is true. Intuitively, this seems true. We frequently define complex properties using this way. Given an open formula, for example, ``$x$ is white and $x$ is big", certainly there is some property $P$ such that necessarily, a thing $y$ has $P$ iff $y$ is white and $y$ is big. In other words, it seems that we should be able to use any formula A to give the necessary and sufficient conditions for something having a certain property. 

Further, we need (Comp_M) to capture compelling natural language inferences, for example the following:\footnote{Timothy Williamson, \textit{Modal Logic as Metaphysics} (Oxford University Press, 2013), p. 227.} 

\begin{quote}
\begin{tabulary}{\textwidth}{lLr}
P$1$. & Alice doesn't smoke a cigar, but she could have done so. & ($\neg Sa \wedge \Diamond Sa$) \\
C. & Alice doesn't do something she could have done. & ( $\exists X (\neg Xa \wedge \Diamond Xa)$ )
\end{tabulary}
\end{quote}
This inference is valid. To capture this, we need precisely an instance of (Comp_M): $\exists X \Box \forall x (Xx \ba \neg Sx \wedge \Diamond Sx)$.\footnote{One might argue that this valid inference can be equally captured by adding an existential generalization axiom to the logic. I just want to point out that this EG axiom is in the exact same spirit as (Comp_M) here: they are both saying that we can form complex properties from simpler ones. So they are not in tension: if one accepts one, one should have reasons to accept the other. } 
Finally, more generally, (Comp_M) is an example of comprehension principles for higher-order logic (even for non-modal logic). Standard second-order non-modal logic usually has comprehension principles of similar form: given any formula $A$, $\exists P \forall x (Px \ba A)$. This ensures that the logic has enough power to prove important theorems  that intuitively needs to be provable. For example, second-order Peano arithmetic typically contain the following Induction axiom: 
\begin{quote}
(Induction Axiom) \hspace{\labelsep}  $\vdash \forall P \forall x (P(\mathbf{0}) \wedge (P(x) \ra P(x+1)) \ra \forall x P(x))$ 
\end{quote}
Now suppose I have the following formula: $x$ is even or $x$ is odd. Certainly, every natural number has this property: being either even or odd. However, the formula itself cannot instantiate the induction axiom given above, as it is a formula not a predicate. With the comprehension principle, we have: $\exists Q \forall x (Qx \ba x $ is even or $ x $ is odd)$ $. Then we can fix on this property $Q$ and use it to instantiate the induction axiom.\footnote{I am using a very informal argument here to motivate and illustrate the use of comprehension principles. For one, ``even" and ``odd" are not primitive in the formal language of arithmetic, but must be defined. For another, the exact proof does not go the way the informal illustration went. However, these are technical details irrelevant for illustrating the use of comprehension principles, so I will not go into them here.}
This shows again that (Comp_M) is not some novel/strange principle that Williamson cooked up but a typical example of logical principles in higher-order logic. So to sum up, (Comp_M) is a very natural and useful logical principle that we want to add to our higher-order modal logic. 

Now we can move on to reconstruct Williamson's challenge. Suppose we instantiate $A$ with $x=y$. We will derive the following:
\begin{quote}
(Haecceity) \hspace{\labelsep} $\Box \forall y \Box \exists X \Box \forall x (Xx \ba x=y)$ \footnote{Necessitation is: if $\vdash A$, then $\vdash \Box A$; Universal Quantifier Rule is: if $\vdash A$, then $\forall x A$. }
\end{quote}
Informally, this says that necessarily everything necessarily has some property such that having this property is necessary and sufficient for being that thing. This property (of necessary and sufficient for being this thing) can be called the haecceity of that individual, following previous literature. 

Next, we can introduce some terminology: let $Haec(X)(y)$ abbreviate $\Box \forall x (Xx \ba x=y)$, informally, ``X is the haecceity of y"; $Tra(X)(y)$ abbreviates $Haec(X)(y) \wedge \neg \Diamond \exists z (Haec(X)(z) \wedge y \neq z)$, informally ``X tracks y". Then, we have: 
\begin{quote}
(Tracking)  \hspace{\labelsep} $\vdash Haec(X)(y) \ra \Box Tra(X)(y))$\footnote{I include a full proof in the Appendix (Section 6), and say more about the significance of the proof.} \smallskip \\
Informally, ``my haecceity necessarily tracks me."
\end{quote}
Then consider an individual $o$ (say, John). By the above theorems, we have: 
\begin{quote}
($o$-Haecceity) \hspace{\labelsep} $\vdash \Box \exists X Haec(X)(o)$\footnote{We here instantiate (Haecceity) with $o$.}
\end{quote}
Putting the above two theorems together, we can derive: 
\begin{quote}
($o$-Tracking)  \hspace{\labelsep} $\Box \exists X Tra(X)(o)$
\end{quote}
Now the challenge according to Williamson is this: 
\begin{quote}
``Even if I had never been, [...], there would still have been a property tracking me (and only me). But how can it lock onto me in my absence? In those circumstances, what makes me rather than something else its target?"\footnote{Williamson, \textit{Modal Logic as Metaphysics}, p. 269.}
\end{quote}
In other words, there is a challenge to contingentists who accept (Comp_M) to explain how the haecceity of an individual can track this individual in a situation where that individual does not even exist. More intuitively, one might identify haecceities ostensively: when John is here, I can point to him, and say \textit{the property of being John, that person}. However, in a case where John does not even exists, how can you identify such a property? How can a property in that situation manage to behave like a haecceity of John? Even if some property manages to do that, what can possibly explain why it necessarily targets this non-existing individual but not some existing individual? What would the identity condition be when comparing an existing individual and non-existing one?  
Put in more formal terms, the contingentist needs to explain why ($o$-tracking) is true while $\Diamond \neg \exists y o=y$. 
The same challenge can be given for anti-haecceity of individuals, the property $X$ such that $\Box \forall x (Xx \ba x \neq y)$. I will not reiterate the argument here. I call this challenge the Explanation Challenge since it is demanding contingentists to offer an explanation of some sort about consequences of their view. 

I will end this section with a final clarification note on the broader dialectical situation in Williamson' book. The above Explanation Challenge is what Williamson deemed as ``the first horn" in a dilemma for contingentists. The ``other horn" is when contingentists attempt to weaken (Comp_M), which is a natural response if one finds the Explanation Challenge a genuine problem. Williamson in the second half of the chapter argued that this weakening also faces serious problems. Thus, the Explanation Challenge is only a part of a larger argument against contingentism. I do not attempt to survey and evaluate the other horn in this paper. 

\section{Two worries about Williamson's challenge}
\subsection{The Minimalist Response}
The minimalist response is motivated by the intuition that there is not really much to explain. In other words, they want to insist that some metaphysical claims do not require substantive explanation. This is a response \emph{on behalf of} contingentists adopting (Comp_M). 
There are two specific strategies implementing this response: (i) insisting that no explanation is required, and this does not render contingentism in a dialectically weaker position; (ii) insisting that there is a trivial explanation, and so contingentism again does not fair worse. I will develop these two strategies in more detail and respond to them on Williamson's behalf. 

Strategy 1 can be developed in the following, Williamson pointed out that individuals can have contingent existence, but there is always a property tracking them. However, this could be seen just as a brute fact of the modal structure of the world and requires no further explanation. Contingentists do not need to be impressed by this phenomenon at all. This response can be generalized to respond to Williamson's challenge concerning the asymmetry between first-order and higher-order necessitism. The challenge is that given higher-order comprehension principles like (Comp_M), one can prove higher-order analogues of first-order necessitism (which is shown in Section 1), like the following second-order version: $\Box \forall X \Box \exists Y \Box \forall x (Xx \ba Yx)$.\footnote{Williamson, p. 264.} Thus, contigentists will need to endorse this systematic asymmetry between first-order and higher-order claims.  
A contingentist can just say that this is exactly what they adopt, and the consequent asymmetry is just a fact that does not call for any further explanation. It is worth emphasizing that this minimalist should not be thought of as ``resisting" or ``refusing" to explain ($o$-tracking), but does not see the need to explain in the first place. 

I find this minimalist response unconvincing. My criticism will be different from Williamson's, so I will not reiterate his arguments here.
First, adopting a minimalist response does not refute or falsify contingentism. It is just that in this dialectical context, contingentism will look much less attractive because there is an alternative theory that has a perfectly simple explanation of ($o$-tracking). More generally, not being able to explain something is no defeat for the theory (probably every theory has something that it has not yet explained), yet in comparing theories, a phenomenon that one theory can readily explain while the other cannot certainly favors one over the other. In this case, necessitism has a very simple explanation of ($o$-tracking): just as the usual case, they can point to $o$ which exists necessarily, and identify the property of \emph{being that thing}. 

Here is an analogy with the Supervenience Challenge to metaethical non-naturalism, the thesis that moral properties are sui generic non-natural properties. The Supervenience Challenge is also an explanatory challenge. The Supervenience Thesis (abbreviated as ``(Supervenience)") of the moral properties on the natural properties claims that two objects cannot differ in their moral properties unless they differ in some natural properties.\footnote{For a more concrete example, we can imagine John and Bill, who are students in the same class. They both arrive at class on time, handed in assignments on time, etc. Now if the teacher start to punish John for alleged moral reasons, he is rightly to object that the teacher's moral assessment is groundless: what could possibly distinguish him from Bill morally? For a more abstract example, one can imagine John in our world and John' in another possible world. Suppose they do exactly the same things and have the same intentions, etc. It seems that they must receive the same moral evaluation (whether that is virtuous or evil): what could possibly distinguish John from John' morally? \\
Here is (Supervenience) formulated in higher-order logic just to draw out the analogy with the current case more clearly: 
\begin{quote}
(ST) \hspace{\labelsep} $\Box \forall X (Moral(X) \ra \forall x (Xx \ra \exists Y (Natural(Y) \wedge Yx \wedge \Box \forall y (Yy \ra Xy))))$
\end{quote}} 
Here the challenge for non-naturalism is to explain why (Supervenience) holds. The key is not the first box since that is usually understood to be conceptual necessity but the second box representing metaphysical necessity.\footnote{Note that contingentists cannot appeal to conceptual necessity or facts of meaning to explain ($o$-tracking) since all of the boxes in the theorems refer to metaphysical necessity. At least as Williamson framed the debate, contingentism and necessitism are full-blown metaphysical theories about the world. I suspect that there are ways to think about this debate using conceptual methods, which will be beyond the scope of this paper but interesting to explore.}
In other words, why the instantiation of natural property necessitate the instantiation of some non-natural property? Just like the minimalist sketched above, some non-naturalists have tried to argue that (Supervenience) does not need an explanation. It is just a fact about the metaphysical structure of the world. This quietist response does not falsify non-naturalism. It just puts non-naturalism in a dialectically weaker position, especially when there are alternative theories which offer an explanation: naturalism does this by identifying moral properties and natural ones. 
The upshot is that the force of the Explanation Challenge does not derive from posing a counterexample/contradicting contingentism but identifying a source of explanatory weakness.
 
Further, I think when phrased in terms of explanation, the burden will be on to contingentism to say why the phenomenon does not demand explanation. Here is a parallel in the sciences. We encounter some natural phenomenon: water freezes in winter, leaves fall down in the fall, etc. The default is that these all call for explanation. The only exception might be that when we get to the most fundamental level of nature:  only when we get to the fundamental particles can we say: those particles just have those properties they have, by nature. There is nothing more we can say. It is simply bad science if one look at a macroscopic phenomenon and just say it is just there and requires no special explanation at all.\footnote{While for daily life/practical purposes, this attitude is entirely justified, it is not for scientific purposes. Otherwise, it is hard to see how explanatory science can ever get started. Consider vision science. One basic question is, what explains our visual capacity? If a person comes along and says ``Well, I can see those things, and not some other things. That's just how I am evolved. What else is to explain?'' That is just bad vision science.}
 The same applies for modal metaphysics as long as it aspires to be (explanatory) science. The default is to assume every modal facts about ordinary objects require explanation, and only when we get to the most fundamental level can we resist the explanation demand. The upshot of all these is that a minimalist adopting strategy 1 will simply be doing bad metaphysics. 
 
Strategy 2, which is more interesting, can be developed in the following way. Contingentists can accept that there is an explanatory demand but argue that there is a trivial explanation. Specifically, contingentists can argue that higher-order necessitism and tracking follows \emph{logically} from $Comp_M$ and the background logic, and that explains why there is tracking and ``locking on to individuals." Again, there is nothing further to it. To take an analogy, suppose one was asked why he believes that $A$ and $B$, he can answer: ``I believe $A$, and I believe $B$, so I believe the logical consequence of my beliefs, namely $A$ and $B$.'' He has indeed provided an explanation of his belief in the conjunction, though a trivial one. Further, the contingentists can even use Williamson's argument in the latter half of the chapter: Williamson discusses how (Comp_M) is the superior comprehension principle in higher-order logic and various technical reasons why one wants to adopt this as part of the logic rather than weaker principles: simplicity, elegance, and enough expressive power to serve logical/metaphysical purposes. Thus, contingentists can even maintain that they have independent reasons to adopt (Comp_M) and argue that ($o$-tracking) is a logical consequence. 

I find this strategy to be problematic, too. 
Firstly, it seems that the logical consequence explanation does not give us a deep-enough explanation. Suppose a contingentist does offer this explanation; a necessitist can just inquire further for an explanation of (Comp_M). Why is that for any condition whatsoever there exists a property such that that condition holds just in the case it is instantiated? Notice now, contingentists cannot use the Williamsonian justification for (Comp_M), since that will answer the wrong question - why we should \emph{believe in/accept} (Comp_M). Thus, the explanatory demand just gets pushed further back. In general, one can always explain $p$ by saying that it is logical entailed by some $q$. But then the explanatory demand just got pushed back to why $q$. Of course, if a theory has to push back infinitely, then it is not a good theory. 

Secondly, I worry that logical consequence is too lax an explanatory basis; that it generates bad explanations. 
Here is an example that I have in mind:

\begin{quote}
(Racist Explanation). \hspace{\labelsep} The Racists believes that every member of race A is evil. Consequently, he believes that a member of A $o$ is evil. When asked why he thinks $o$ is evil and consequently refused to offer $o$ equal payment/respect as other employees, the Racist says, ``Well, this is a logical consequence of my belief. What more do you want me to explain?''
\end{quote}

There is something wrong with both explanations.\footnote{Of course the Racist is subject to my previous challenge as well: I can simply ask the Racist why he believes that every member of race A is evil in the first place. He is not discharged of explanatory demand. But here I am developing a different problem for the Racist.} 
One, what validates the Racist's universal belief is precisely the character of each individual. Thus, the Racist is not entitled to appeal to this universal belief as an explanation of his specific belief. Two, the Racist's belief attributes some structure to human beings in general and to particular individuals like $o$. He cannot simply ignore this structure by appealing to the general principle that attributes this structure. He needs to look at this specific person and explain why the ``evil" structure is really there, as his theory claims it is. 
One can now see the same problems hold for contingentism too: (i) ($o$-tracking) is supposed to support (Comp_M), not the other round. Whether contingentism should accept (Comp_M) partially depends on whether they find its consequences compelling. The logical consequence explanation gets the order of explanation wrong; (ii) contingentists by adopting (Comp_m) attribute some metaphysical structure to individuals (like ($o$-tracking)), and they now need to find actual features of the world/individuals that support this structure in order to vindicate their theory. In either case, contingentists are not done.
To sum up, logical consequence cannot serve as a good explanation. Consequently, Strategy 2 fails.\footnote{An anonymous reviewer asked whether there are previous contingentists literature defending the worry that I criticized in this section. The reviewer asked because my criticism of this contingentist response seems sweeping, and so this response may seem like a ``low hanging fruit objection that contingentist philosophers would, in general, avoid". There are several things to say: \\
Firstly, the relevant literature does not seem to have focused particularly on the Explanation Challenge and the role of explanation. 
Second, although my objections can seem comprehensive, I think they only scratch the surface of the relevant problems related to explanation, the need for explanation, etc. I am sure there are insights that contingentists can bring in from the literature about explanation in general (and philosophy of science etc) to resist my arguments. However, my goal in this paper is to articulate more clearly a worry that was raised in the seminar discussing this book and give some compelling response to it. So I did not get into potential larger debates about explanation and related notions.}

\subsection{A worry about the notion of explanation}
The following worry is not so much directly on behalf of contingentists but trying to undercut Williamson's entitlement to raise the Explanation Challenge in the first place. 

There are two parts to this worry. 
Firstly, one might be skeptical of the notion of explanation evoked here. At a first pass, the explanation demanded seems metaphysical.\footnote{It certainly is not causal or physical: there is no causal relationship or ``physical" process in place. It is not constitutive either: nothing seems to constitute another. Normative explanation is certainly not what is being demanded: they are descriptive claims through and through.} However, Williamson is skeptical of notions like grounding and truth-making (the often-evoked notion in metaphysical explanation), so they cannot be used to give metaphysical explanation. Thus, how is Williamson's demand for a metaphysical explanation (potentially using these notions) from the contingentists legitimate? To put it in another way, suppose Williamson thinks that the notion of ``metaphysical" explanation (as distinct from physical/causal, social, mathematical, etc) is bogus, and there is no genuine metaphysical explanation at all. Then, trivially, there is no genuine metaphysical explanation for the truth of ($o$-tracking) and (Contingentism). So how can he demand contingentists perform this task? Perhaps more strongly, he should actually be happy with contingentists giving no metaphysical explanation as that is exactly what they should be doing. 

Secondly, one might be skeptical of whether Williamson is entitled to this challenge given his methodology. Williamson is very insistent on evaluating metaphysical theories by appealing to their simplicity, strength, and elegance. One of the main threads of his book argues that necessitism allows us to have much simpler semantics for quantified modal logic by allowing us to treat possible world models realistically, adopt simple elegant axioms, avoid complex restrictions in the proof theory, and have an expressive high-order language. Those notions like simplicity/strength are easily demonstrated by observing the kind of formal apparatus present. In contrast, the notion of explanation is comparatively murky.

I think both worries can be dissolved. 
Regarding the first worry, Williamson can respond in the following ways. 
One is that we do not have to fix on the nature of explanation prior to giving one. There is an intuitive grasp of explanation that one can rely on. Consider the Supervenience Challenge again. There, a ``metaphysical" explanation is also demanded. But one does not necessarily have to give a ``metaphysical" explanation in terms of grounding, metaphysical laws, essences, truth-making, etc. Naturalists just say that moral properties are identical to natural ones, so (Supervenience) is a trivial truth; expressivists just say that I cannot explain (Supervenience) as it is about ``inflated" properties, but I can explain why people have good reasons to commit to (Supervenience).\footnote{Gibbard had given one such account. Now whether expressivists are thereby discharged is actually not trivial. See Dreier, 2015. Dreier argues that quasi-realist formulation of expressivism, which one may reject, is not off the hook. However, this is beyond the scope of this paper.} Those explanations are in no way ``metaphysical". Similarly, the explanation of ($o$-tracking) does not need to end up being distinctively metaphysical.\footnote{The necessitists' explanation is not metaphysical at all. One can just imagine the counterfactual circumstance and point to the existing object, concrete or not, and say the property of being that object.} 
In fact, it is actually the \emph{contingentists} for whom we might have this kind of worry about problematic ``metaphysical" explanations, because they presumably need to give some ``metaphysical" explanations for ($o$-tracking), just like one might be skeptical of non-naturalism when non-naturalists invoke all kinds of metaphysical notions/relations (normative laws, grounding, normative essences, hybrid properties, etc) in their explanation of (Supervenience).\footnote{To illustrate, Stephanie Leary posited hybrid properties that have both normative and natural essences to explain (Supervenience) --- see Leary, 2017. In general, I think that her additional ontology is not independently motivated and appears very ad hoc as it does not handle any other major (explanatory) challenges to non-naturalism like explaining our moral knowledge or making sense of alien communities and their ``moral" language. Of course, these objections need to be further elaborated and defended, which is beyond the scope of this paper. The point I am making here is that one can \emph{easily} come to worry about whether these metaphysical entities/postulates are real, whether they are really motivated, whether they really explain anything, etc. The upshot is that these (potential) worries about non-naturalist evoking heavy-duty metaphysical notions are exactly parallel to worries about contingentists evoking heavy-duty metaphysical notions.} Thus, in fact, having to recourse to inflated metaphysical explanation is a disadvantage for contingentists, not for Williamson. 

Regarding the second worry, I think the best response is to argue that theory comparison cannot \emph{all} be about simplicity, strength, and elegance especially when the theory is about the actual world. This is true for scientific theories. Comparison between physical theories cannot all come down to which one has more simplicity, strength, and elegance. These formal criterion are very important and rules out many strange theories, but they cannot be all there is to it. Physicists look at fit with empirical data, explanatory power, etc. One cannot decide between Newtonian mechanics or Relativisitic physics just on their formal features. 
Here is a more specific version for contingentism vs. necessitism. One cannot just decide between them based on formal features. Take the argument that necessitism provides a simpler formal semantics for quantified modal logic. Whether this is true depends on what language one is speaking. Suppose one is speaking a contingentist-commiting language, then of course contingentist semantics will be a better fit and simpler.\footnote{To see the challenge: consider religious discourse. There is this word ``God" that appear in the discourse very often. What is its meaning? In some sense, a semantics that assigns it a supernatural being is the simplest as it will validate all the sentences in which the word occur, compared to a semantics that does not. But of course, this cannot be an argument for the existence of ``God".} A similar reverse argument applies to logical strength. Contingentists might say if you adopt all those principles, it will vastly over-generate statements that are false or suspicious so one has to block their derivation. That surely does not look elegant. The upshot is that for empirical theories, we need to look at their empirical content and whether that fits with the world in the right way. 
Thus, Williamson need not only appeal to formal notions to adjudicate between theories. He is completely entitled to appeal to explanatory power and fit with data to evaluate the two theories. This accords with Williamson's general methodology in any case: modal metaphysics is a science. 

\section{An Anti-Haecceitist response}
In this section, I explore a substantive contingentist response to Williamson's challenge. I argue that if true, this successfully answers Williamson's challenge, but this response ties contingentism to another controversial metaphysical doctrine. 

The response is to first adopt anti-haecceitism and then use that to explain ($o$-tracking) even when $o$ does not exist. Anti-haecceitism (abbreviated as AH) is roughly the thesis that there are only qualitative properties (consequently all seemingly non-qualitative properties are reducible to qualitative ones). Qualitative properties are everyday familiar properties like volume/size, color, material composition, or relational properties (\emph{being the mother of}, etc).\footnote{Robert Stalnaker, \textit{Mere Possibilities: Metaphysical Foundations of Modal Semantics} (Princeton University Press, 2012).} Non-qualitative properties are harder to describe directly for they are intended to pick out precisely those properties that are not qualitative. Perhaps the most intuitive example is the property of \emph{being this very object}.\footnote{I think it is hard to identify non-qualitative properties in an entirely theory-neutral way, since AH precisely denies their existence! However, generally, non-qualitative properties are motivated by thought experiments like the following: Consider a possible world where there are only two balls. They are identical in every aspect: shape, size, color, material, etc. Suppose one brings in spatial locations or relationships to an observer, then one is ``illegally" bringing in distinct qualitative properties or assuming the existence of an observer that the thought experiment stipulated not to exist. Thus, there seems to be no way of distinguishing between them but through properties like \emph{being this very ball} and \emph{being that very ball}. }
The point of debate is whether for a particular individual $o$, the property of being $o$ is qualitative or not. AH says yes, haecceitists say no. In other words, AH maintains that this property of \emph{being $o$} can be reduced to a complex qualitative property like the property of \emph{being the individual that occupies this location, has this height/weight/taste/wealth, ...}. 
Now we can see the AH response to Williamson's challenge. Suppose that $o$'s haecceity $X$ can be reduced to this complex qualitative property built out of simple qualitative properties. Suppose further that $o$ does not exist in some counterfactual world. In this situation, very plausibly those simple qualitative properties still exist and so does this complex property. Then indeed there will be a haecceity of $o$, namely this complex property, and it locks on to $o$ because by stipulation whatever instantiates it will be identical $o$ instead of identical to other objetcs. $o$ is ``defined" by all and only these properties.\footnote{Here $o$ does not have to be identified with those properties. So the AH explanation does not have to commit to a bundle theory of individuals.} Note that $o$ does not have to exist: this complex qualitative property is just not instantiated in this situation. So AH gives contingentism a \emph{substantive} explanation of why $o$ can fail to exist while ($o$-tracking) is still true. 

One might be tempted to say that AH targets the wrong explanandum. The idea is that AH response is phrased in terms of properties and individuals instantiating them, but Williamson's (Comp_M) and ($o$-tracking) are all formulated in higher-order logic. Further, if one thinks that higher-order quantification is not first-order quantification over objects of higher-type, then the AH response has failed to explain the right thing. They have explained why properties of a certain kind track the individual, but they have not explained why $\Box \exists X Tra(X)(o)$ given $\Diamond \neg \exists y o=y$. 
However, this response is not successful since one can easily formulate AH in higher-order logic: 
(AH) $\forall x \forall X (Haec(X)(x) \ra Q(X)$ where $Q(X) := $X is qualitative$ $. \\
Then one can reformulate the AH response above in those terms. 
In fact, after the formulation in higher-order logic, there can be a very clear path to a successful explanation. Suppose one has discovered that: \begin{quote}
$\Box \forall x (x=o \ba P_1x \wedge P_2x \cdots \wedge P_nx)$ where $P_i$ is a qualitative property for any $i$. \end{quote}
Then one can form a complex predicate $P_{total}$ such that: 
\begin{quote}
$\Box \forall x(x=o \ba P_{total}x)$ where $P_{total} := \lam x. P_1x \wedge P_2x \cdots \wedge P_nx$. \end{quote}
Then one can easily, by Existential Generalization, derive:
\begin{quote}
$\exists X \Box \forall x(x=o \ba Xx)$ \end{quote}

Williamson has various responses to his anticipated AH response. Against a purely qualitative conception of properties, he said: 
\begin{quote}
``Thus the purely qualitative conception of properties may well require a highly contentious form of the identity of indiscernibles for individuals, on which qualitative identity entails numerical identity. That is a far less plausible claim than the trivial form of the identity of indiscernibles that permits non-qualitative properties such as identity with y. We have no serious evidence against the metaphysical possibility of a symmetrical universe in which every individual can be reflected (rotated, translated) onto its qualitative double."\footnote{Williamson, \textit{Modal Logic as Metaphysics}, pp. 271--272.}
\end{quote}
However, I think this response targets AH as a self-standing doctrine rather than the AH explanation of the Explanation Challenge, which I will come back to. 

Against the previously developed AH response, he said: 
\begin{quote}
``The theory becomes still more elaborate once fitted out with an account of the persistence of individuals across times and possibilities, since an individual typically has many of its purely qualitative properties, such as shape and size, temporarily and contingently. Alternatively, if the theory denies identity through change and through contingency, not only is that yet another implausible consequence, which requires still more theoretical complexity to save the appearances, it also fits badly with the underlying motivation for contingentism, by treating a vast range of apparent contingency as an illusion. Thus the purely qualitative conception drags the contingentist into proliferating complications of metaphysical theory with no independent plausibility.''\footnote{Williamson, p. 272.}
\end{quote}
I find the remarks here sketchy, too. 
For one, I am not sure if AH should regard having contingent qualitative properties as a problem for their theory. Either they can say that the qualitative properties include modal and temporal properties (so \emph{being John} involves \emph{being a possible lawyer}) or they can index haecceities such that one should really say: $X$ is a haecceity of $y$ at $w,t$. In this second way, the original theorems are still in place because, in any world $X$ is still a haecceity of $y$ at a particular indice $w,t$.\footnote{Now I do not want to get into the debate between AH and haecceitis here. These two ways of responding to Williamson's challenge at least seem prima facie available. They could be false in the end, but at least they show AH is not easily defeated. } 
Further, and more importantly, I do not think his challenge shows that the AH explanation is unsuccessful. Williamson at best shows that AH has various undesirable consequences.
Thus, overall, I think Williamson's own remarks in the chapter do not constitute the right kind of dialectic challenge to the AH response.  

However, Williamson's skepticism brings us to what I think is a more successful general response to the AH explanation. The idea is to admit that the AH strategy does respond to Williamson's challenge, and to point out that this AH response forces contingentism to accept a controversial metaphysical doctrine, thereby having to accept its dialectical challenges. 

Firstly, adopting the AH strategy forces contingentists to be anti-haecceitists, thereby having to answer challenges to anti-haecceitism itself. 
Contingentists need to accept AH to give the AH explanation. Granting that AH does offer a successful explanation of ($o$-tracking), we can say that
if AH is indeed true, then we can solve Williamson's Challenge. Consequently, the plausibility of the theory package consisting of solving Williamson's Challenge and AH - will be the plausibility of AH itself. Assuming that Solving Williamson's Challenge amounts to vindicating contingentism, the plausibility of this version of contingentism (which will be the conjunction of AH and (Contingentism)) will be the plausibility of AH itself.\footnote{Readers familiar with probability theory might think in terms of probabilities assigned to these propositions: the conditional probability P(Solving Williamson's Challenge $|$ AH) is roughly 1. So P(Solving Williamson's Challenge $\wedge$ AH) will be equal to P(AH). So  P(Contingentism $\wedge$ AH) = P(AH). These follow from basic probability axioms and logic. }
Now there is a problem if AH is not very plausible. If AH is not a compelling metaphysical thesis in the first place, contingentism will not fare well having to accept it.
This is where many of Williamson's previous charges can be properly incorporated: skepticism about distinction between qualitative and non-qualitative properties, counterexamples from indiscernible objects, etc. The upshot is that now the dialectic cost for contingentism is no longer not being able to explain something but having to accept some controversial/implausible doctrine in order to be able to explain something. This seems to be a cost. Contingentists give themselves a greater burden compared to a necessitist that can remain neutral on this issue. 

I am personally not very worried about combining a theory with another controversial theory in itself. Intuitively, theories should be allowed to appeal to other resources (like other theories) in developing and defending itself even if those resources are controversial. Denying the legitimacy of this appeal would render theorizing very difficult and limited. We want to establish connection between theories across domains and explore how they can inform each other.\footnote{I can give numerous examples. For example, expressivists in metaethics appeals to truth minimalism to recover the legitimacy of ordinary moral talk/thought, even if truth minimalism is controversial; non-naturalists appeals to post-modal/hyperintensional metaphysics in developing their theories, even if notions like grounding/essence invoked in hyperintensional metaphysics are very controversial --- see Bengson, Cuneo, and Shafer-Landau, 2024. I think they can make these appeals. Metaethicists have made a lot of progress by doing this. Their theorizing would just be very limited if they cannot do this. }
However, the real worry is whether this combination is the only viable combination (because we do not know which theory is ultimately right). That is, if contingentism can only be effectively defended relying on a particular theory of haecceities, then it looks less attractive than a view that is compatible with a variety of theories of haecceities. 
Contingentism is not supposed to be a global thesis that aims to provide complete answers to all metaphysical questions. It is not even aiming to be a comprehensive theory of metaphysical modality. Necessitism is the same. Thus, one would hope that it can remain local instead of having global consequences. However, if it can only be a good local theory when combining with a particular (global) theory, then one should be more skeptical as this local theory seems to demand too much packaged in along with it. Necessitism in contrast is compatible with both haecceitism and AH.\footnote{This is exactly the same for expressivism and truth minimalism. Expressivism is meant to be a local thesis about moral language. However, to defend it, one would need to reject truth-conditional semantics (which is incompatible with truth minimalism) in general, then it no longer seems very attractive.}
Thus, overall, while AH response is a good substantive explanation answering the Explanation Challenge, there will be considerable dialectical cost for contingentists to accept it.  

\section{Conclusion}
In this paper, I have examined three responses to Williamson's Explanation Challenge and argued that each response faces their own problems. While I argue for the stronger conclusion that the first two challenges fail, I argue for the weaker conclusion that the last response succeeds but only with additional dialectical cost to contingentism. I hope this paper has helped to clarify the stake of Williamson's ``first horn" to contingentism in Chapter 6 and strengthen his argument against contingentism. 

\section{Appendix}

\subsection{The proof for (Tracking)}
First, we can observe the following proof: \\
\begin{quote}
\begin{tabulary}{\textwidth}{Lr}
$\forall x (Xx \ba x=y), Haec(X)(z) \wedge z \neq y \vdash \bot \ $ & (Reductio, Cond. proof, Universal generalization)  \\
\end{tabulary}
\begin{tabulary}{\textwidth}{Lr}
$\vdash \Box \forall x (Xx \ba x=y) \ra \Box \forall z \neg (Haec(X)(z) \wedge z \neq y)$ & (K) \\
$\vdash \Box \forall x (Xx \ba x=y) \ra \neg \Diamond \exists z (Haec(X)(z) \wedge z \neq y)$ & (Equivalence) \\
$\vdash Haec(X)(y) \ra Tra(X)(y)$ & \\
\end{tabulary}
\end{quote}
Now I will show the first line. \\
\begin{quote}
$\forall x (Xx \ba x=y), Haec(X)(z) \wedge z \neq y \vdash Xy \ba y=y \\
\forall x (Xx \ba x=y), Haec(X)(z) \wedge z \neq y \vdash Xy \ba z=y \\
\forall x (Xx \ba x=y), Haec(X)(z) \wedge z \neq y \vdash y=y \ba z=y \\
\forall x (Xx \ba x=y), Haec(X)(z) \wedge z \neq y \vdash z=y \\
\forall x (Xx \ba x=y), Haec(X)(z) \wedge z \neq y \vdash z \neq y \\
\forall x (Xx \ba x=y), Haec(X)(z) \wedge z \neq y \vdash \bot $ \\
\end{quote}
Then, from $\vdash Haec(X)(y) \ra Tra(X)(y)$, we can observe that: \\
\begin{quote}
$\vdash \Box Haec(X)(y) \ra \Box Tra(X)(y) \ \ \ \ $(K) $ \\
\vdash \Box \forall x (Xx \ba x=y) \ra \Box \Box \forall x (Xx \ba x=y) \ \ \ \ $(\textbf{4})$\\
\vdash Haec(X)(y) \ra \Box Haec(X)(y) \ \ \ \ $(Chaining conditionals)$ \\
\vdash Haec(X)(y) \ra \Box Tra(X)(y)$ 
\end{quote}
\subsection{The proof for (o-Tracking)} 

\begin{quote}
$\vdash Haec(X)(o) \ra Tra(X)(o)$ \ \ \ \ (Proved above) \\
$\vdash \exists X Haec(X)(o) \ra \exists X Tra(X)(o)$ \ \ \ \ (Derivable from $\forall$ rule) \\
$\vdash \Box \exists X Haec(X)(o) \ra \Box \exists X Tra(X)(o)$ \ \ \ \ (K)\\
$\vdash \Box \exists X Tra(X)(o)$ \ \ \ \ (MP, o-Haecceity) 
\end{quote}


\noindent I include these proofs in detail for two reasons. One, Williamson did not lay out the proof at all in the book. So I think reconstructing it here will help the reader to see clearly how the seemingly strong principles are derived. Second, and more importantly, this proof shows how little background logic is needed to derive the later-shown-to-be-problematic (Tracking). This proof assumes only modal logic principles \textbf{4} and \textbf{K}, and the usual meta-rules like conditional proof, reductio, etc. Thus, it  does not require a strong logic to prove (Tracking). The significance is that, suppose one accepts that (Tracking) has problematic consequences, one thing we can always see is if there is any logical principle we can reject which contributes to the proof. That would be a natural contingentist way out. However, this proof shows that it will not be easy to take this route. K is the least contentious axiom in modal logic; \textbf{4} is somewhat controversial, but not very, since intuitively, modal properties/facts should themselves be necessary and not mere accidental. Further, the controversial B axiom that actually bears on the necessitism and contingentists debates are not used essentially. Thus, there is not much reasonable/non-ad hoc room for contingentists to weaken their background logic to escape from Williamson's challenge. Williamson himself does not make this point, but I think it is important. 

\refsection

\begin{hangparas}{\hangingindent}{1}
  Bengson, John, Terence Cuneo, and Russ Shafer-Landau. \textit{The Moral Universe.} Oxford University Press, 2024.

  Dreier, James. ``Explaining the Quasi-Real.'' \textit{Oxford Studies in Metaethics} 12 (2017): 273--297.

  Leary, Stephanie. ``Non-Naturalism and Normative Necessities.'' \textit{Oxford Studies in Metaethics} 12 (2017): 76--105.

  Stalnaker, Robert. \textit{Mere Possibilities: Metaphysical Foundations of Modal Semantics}. Princeton University Press, 2012.

  Williamson, Timothy. \textit{Modal Logic as Metaphysics}. Oxford University Press, 2013.
  \end{hangparas}
%%% Local Variables:
%%% mode: LaTeX
%%% TeX-master: "../main"
%%% End:
