
\chapter{Selfish Comparative Optimism: A Rejoinder to Nagasawa's \emph{Problem of Evil for Atheists}}
\chaptermark{Selfish Comparative Optimism}
\chapterauthor{Wilson Sugeng,
\textit{University of St Andrews}}


% Makes the section headings be formatted so it does `Section 1:', instead of just '1'
\renewcommand*{\thesection}{Section~\arabic{section}:} 
% \thesubsection might use \thesection, therefore it is also redefined
\renewcommand*{\thesubsection}{\arabic{section}.\arabic{subsection}}


\begin{quote}
Yujin Nagasawa's problem of systemic evil (POSE) argues that systemic
evils like natural selection pose a greater challenge to
atheism/non-theism than to theism, as they conflict with ``modest
optimism'': the view that the world is fundamentally ``not bad.''
Nagasawa suggests theism resolves this by appealing to a heavenly bliss,
offsetting natural evils, a strategy unavailable to
atheists/non-theists. However, I argue that atheists/non-theists are
better equipped to address POSE because they are not constrained by the
theistic commitment to a categorically good world.

In Section $1$, I critique two theistic approaches to POSE. Extreme
optimism defends the actual world as the best possible one, requiring
problematic justifications such as free-will and ``only-way'' theodicies
to explain systemic evils as necessary. Neutral optimism, while allowing
for multiple good worlds, still struggles to reconcile systemic evils
with a benevolent God, merely shifting the problem to other possible
worlds.

In Section $2$, I explore how atheists/non-theists can bypass POSE. They
can adopt personal, rather than cosmic, optimism, valuing their own
existence without affirming the world's overall goodness. Alternatively,
they can embrace comparative optimism, viewing existence as better than
non-existence without attributing intrinsic value to natural processes
like evolution. These flexible approaches free non-theists from the
philosophical burdens tied to systemic evils.

In Section $3$, I argue that even if POSE persists, atheists/non-theists
can ``borrow'' theists' theodicies without committing to their
metaphysical assumptions. By adopting naturalistic or subjective
frameworks, non-theists can justify their modest optimism without the
theological constraints imposed by theism. This demonstrates that POSE
ultimately challenges theistic frameworks more than atheistic ones.
\end{quote}

\vspace{\credgap}

\section*{Introduction}

In \emph{The Problem of Evil for Atheists,} Yujin Nagasawa develops a
problem of systemic evil (POSE) that he claims challenges both
atheists/non-theists and theists alike.\footnote{When I say, ``God'' and
  ``Theism'' in this paper, I assume an omniscient, omnipotent, and
  omnibenevolent singular/simple creator.} He identifies a tension
between two widely held theses: 

\begin{enumerate}[leftmargin=42]
\def\labelenumi{(\arabic{enumi})}
\item
  Systemic evil: The process of natural selection necessitates
  significant suffering and pain for countless sentient animals.
\item
  Modest optimism: Overall and fundamentally, the environment in which
  we exist is not bad.\footnote{Yujin Nagasawa, \emph{The Problem of
    Evil for Atheists} (Oxford University Press, 2024), 133, 140.}
\end{enumerate}

\noindent While theists naturally affirm modest optimism due to their belief in a
benevolent creator God, Nagasawa observes that atheists/non-theists are
also generally grateful for their existence.\footnote{Nagasawa,
  \emph{The Problem of Evil for Atheists}, 161.} For instance, popular
atheist Richard Dawkins suggests that contemplation of the law-like
evolutionary processes behind our existence puts us ``in a position to
give thanks for our luck in being here''---not a gratitude directed
towards any agent or being, but rather a ``gratitude in a
vacuum.''\footnote{Richard Dawkins, ``The Greatest Show on Earth
  Live'' (lecture, University of Auckland, Auckland, New Zealand, 13
  March 2010).} Nagasawa sees this as inconsistent: expressing
existential gratitude without acknowledging the systemic evils
underpinning it implies a tacit endorsement of these evils.

To illustrate this tension, Nagasawa adapts Janna Thompson's apology
paradox, which holds that regretting an unjust historical event can be
problematic if one's existence depends on that event. For example, a Jew
whose grandparents met during the Holocaust faces a paradox: to regret
the Holocaust may seem to imply regretting her own existence.\footnote{Janna
  Thompson, ``The Apology Paradox,'' \emph{The Philosophical Quarterly}
  50\emph{,} No. 201 (2000): 471.} Thompson resolves this by
distinguishing between regretting \emph{how} one came to exist and
\emph{that} one exists---the Jew can regret \emph{how} her grandparents
met, without regretting \emph{that} they met at all.\footnote{Janna
  Thompson, ``The Apology Paradox,'' \emph{The Philosophical Quarterly}
  50\emph{,} No. 201 (2000): 475.} Applied to POSE, this seems to suggest that one can regret the
mechanisms of natural selection without regretting the outcome of our
existence.

However, Nagasawa argues that this resolution fails in the context of
POSE. Unlike historical events, natural selection is not a contingent
circumstance but a fundamental feature of the natural world.\footnote{Nagasawa,
  \emph{The Problem of Evil for Atheists}, 167.} To reject it is not to
regret a particular pathway to existence, but to undermine the very
conditions that make existence possible. That is, there is no possible
world where natural selection does not govern nature and beings like us
still exist.

Theists, Nagasawa argues, are better positioned to defend modest
optimism, drawing on ``heavenly bliss'' theodicies that justify or
outweigh earthly suffering with the promise of an afterlife. These come
in two forms: (1) as a deferred justification, where evolution is
acceptable because it leads to eternal reward, and (2) as a utilitarian
offset, where infinite heavenly bliss outweighs finite worldly
suffering. Because atheists cannot appeal to such concepts, POSE, he
claims, presents a more serious problem for atheists.

Contrary to Nagasawa, I argue that atheists and non-theists are better
positioned to address POSE because they are not constrained by the
theistic requirement to see the world as overall categorically good. To
support this claim, I first critique two theistic attempts at resolving
systemic evil, namely extreme and neutral optimism, illustrating their
shortcomings. Subsequently, I explore how atheists/non-theists might
effectively sidestep POSE by adopting personal rather than cosmic
optimism, or by embracing a comparative optimism which sees existence as
preferable to non-existence without categorically endorsing the systems
that facilitated it. Finally, I turn Nagasawa's borrowing argument
around to propose that, even if POSE remains challenging,
atheists/non-theists can strategically adopt theistic theodicies without
their accompanying metaphysical assumptions, thereby reducing POSE's
impact and revealing it to be ultimately a greater challenge for
theistic frameworks than for atheistic or non-theistic ones.

\section{Two Theist Modest Optimists}
\subsection{Extreme optimism}

The first theist modest optimists---extreme optimists---claim that
because God actualised the best among all possible worlds, systemic evil
must necessarily exist in all good worlds. Although Gottfried Wilhelm
Leibniz does not himself discuss systemic evil and predates evolution,
his \emph{Theodicy} (1710) presents a system where given God's
omnibenevolence and omniscience---if a possible world is better than the
actual, then God would either not be good enough to desire the best for
the world, or ignorant in not knowing which world is the
best.\footnote{G. W. Leibniz, \emph{Theodicy,} edited by Austin Farrer,
  translated by E. M. Huggard (Open Court Publishing Company, 1985),
  249.}

As an implication, extreme optimists must affirm Nagasawa's claim that
no possible world exists in which natural selection does not govern
nature; for if God is necessary, then no other world is possible.
Natural selection must therefore serve an instrumental role in the
world's goodness. Building on this system, Austin Farrer argues that the
removal of any such purported evil systems will undermine God's
mechanism for bringing about the best world. The goodness of a physical
system, for instance, inherently includes the potential for mutual
interference, leading to evils like predation. Without this
interference---if this world were a ``magically self-arranged garden''
free of competition for space or resources---physicality itself ceases
to exist.\footnote{Austin Farrer, \emph{Love Almighty and Ills
  Unlimited} (Collins, 1962), 53-54.} Removing such systems would
be akin to relieving an animal's pain ``by the removal of its nervous
system; that is to say, of its animality.''\footnote{Austin Farrer, \emph{Love Almighty and Ills Unlimited} (Collins, 1962), 51.}
Regretting natural selection thus implicitly challenges God's
rationality and goodness in creating us as physical beings rather than
spiritual entities.\footnote{Austin Farrer, \emph{Love Almighty and Ills
  Unlimited} (Collins, 1962), 67.}

An immediate difficulty with extreme optimism is that claiming this
world to be the best possible one is hard to reconcile with the presence
of seemingly avoidable evils observed throughout nature. This tension is
captured ironically in the eponymous character of Voltaire's
\emph{Candide} (1759) who insists that this is the best possible world
as he faces a world plagued with wars, earthquakes, and
slavery.\footnote{Nagasawa, \emph{The Problem of Evil for Atheists,}
  129.} Or when Darwin questions why God permitted the creation of the
Ichneumonidae who brutally feeds inside the living bodies of
caterpillars.\footnote{Charles Darwin, ``22 May 1860 Letter to Asa
  Gray,'' Darwin Correspondence Project, accessed on 5 December 2024,
  https://www.darwinproject.ac.uk/letter/DCP-LETT-2814.xml.} This
presents a major challenge: extreme optimism struggles to align with
observable, avoidable evils unless it denies these empirical
observations---as some Creationists do---or reinterprets such systemic
evils as necessary.\footnote{Paul Prescott, ``The Secular Problem of
  Evil: An Essay in Analytic Existentialism,'' \emph{Religious Studies}
  57 (2021): 102.}

Granting natural selection's empirical truth, theists generally present
two kinds of theodicies for \emph{why} God actualised natural selection.
Firstly, theists have adapted the free-will theodicy to address some
non-agential non-human suffering. In traditional free-will theodicies,
God permits agents the capacity to choose evil over good as the goodness
of human agency outweighs the risks of their choosing evil. In one
adaptation, Richard Swinburne argues that animal pain and suffering
exists as examples of evil actions humans can inflict on each other.
Predation therefore exists as an educational tool for humans to observe
and understand how to commit evil, thereby enabling their capacity for
moral choice.\footnote{Richard Swinburne, ``Natural Evil,''
  \emph{American Philosophical Quarterly} 15\emph{,} No. 4 (1978): 299.}

Secondly, theists have adapted a variation of the soul-making theodicy
known as the ``only-way'' theodicy, arguing that certain natural goods
can only develop through natural selection. Holmes Rolston observes that
the predator-prey cycle is instrumental to the beautiful diversity of
animals, where ``The cougar's fang has carved the limbs of the
fleet-footed deer, and vice versa.''\footnote{Holmes Rolston III,
  \emph{Science and Religion: A Critical Survey} (London: Templeton
  Foundation press, 2006), 134.} While Young-Earth Creationism may have
created this diversity instantaneously, Christopher Southgate argues
that natural selection is the only way creatures can develop into
biological ``selves'' with their own interests and
behaviours.\footnote{Southgate, \emph{The Groaning of Creation}, 58.}
This offsets any evolutionary evils for it culminates into complex
``selves'' that conform to God's image.\footnote{Southgate, \emph{The Groaning of Creation}, 72.} This
``selving'' must come independently, for Peter van Inwagen argues that
an irregular world is a defect: God who constantly intervenes and
violates his own laws is either a irrational or evil.\footnote{Peter van
  Inwagen, ``The Problem of Evil, the Problem of Air, and the Problem of
  Silence,'' \emph{Philosophical Perspectives} 5 (1991): 143-45.} So,
common to both free-will and ``only-way'' theodicies is a notion that
some ultimate good offsets the evils of natural selection as an
instrument.

However, these two theodicies only defer the problem of evil to another
system underlying the challenged system. For instance, free-will
theodicies must still address Pierre Bayle's objection: If God's
omniscience foresees that giving humanity free will inevitably results
in unrighteousness, then God is either reckless or cruel to ``gift''
humanity agency, knowing it would lead to their harm and judgment under
his wrath.\footnote{Pierre Bayle, \emph{Historical and Critical
  Dictionary: Selections,} translated by Richard H. Popkin and Craig
  Brush (Hackett, 1991), 177.} Echoing Bayle, Robert John Russell
questions, ``Why did God choose to create \emph{this} universe with
\emph{these} laws of physics knowing that they would not only make
Darwinian evolution unavoidable, and with it the sweep of natural evil
in the biological realm?''.\footnote{Robert John Russell, ``Natural
  Theodicy in an Evolutionary Context,'' in \emph{Cosmology: From Alpha
  to Omega} (Fortress Press, 2008), 259.} It appears, then, that extreme
optimism is burdened with regressive manifestations of the problem of
evil.

In sum, while extreme optimists attempt to reconcile systemic evil with
the claim that this is the best possible world through the use of
free-will and ``only-way'' theodicies, such strategies ultimately defer
rather than resolve the problem. Faced with empirical evidence of
seemingly gratuitous suffering, they must either deny these realities or
accept increasingly speculative theological explanations. While extreme
optimism may appeal to the heavenly bliss defence, it still does not
explain \emph{why} natural selection is the best possible means towards
that end without returning to this regress or begging the question. As
such, extreme optimism appears ill-equipped to resolve the tension
Nagasawa identifies between systemic evil and modest optimism. So,
theists must either concede that natural selection is not the best
necessary instrument in the best possible world, or following Bayle and
Russell accept the former's pessimism or latter's ``agnostic cosmic
theodicy'' in accepting that POSE cannot be answered.\footnote{Robert John Russell, ``Natural Theodicy in an Evolutionary Context,'' in \emph{Cosmology: From Alpha
  to Omega} (Fortress Press, 2008), 255.}


\subsection{Neutral optimism}


The second theist modest optimists, the neutral optimists, reject that
the actual world is necessarily the best, but rather affirms that God
actualised one of many possible overall good worlds. For instance,
Robert Merrihew Adams argues that extreme optimism inappropriately
imposes a utilitarian standard of moral goodness to God's
omnibenevolence. Instead, he argues that traditional Judeo-Christian
ethics account for God's goodness in terms of his grace---an inclination
to love that is not based on the merit of the one being
loved.\footnote{Robert Merrihew Adams, ``Must God Create the Best?'',
  \emph{Philosophical Review} 81 (1972): 324.} Indeed, core to Abrahamic
monotheism is an affirmation of God's aseity, his self-sufficiency and
independence from any external cause or necessity. \footnote{Ian A.
  McFarland, \emph{From Nothing: A Theology of Creation} (Westminster
  John Knox Press, 2014), 61.} If God were obligated to create the best
possible world in order to express his power or love, then his
omnipotence and omnibenevolence would become contingent on something
external---namely, the existence of that world---thereby undermining his
aseity. It follows, therefore, that a being who never exists is not
wronged by not being created, since existence itself is not owed to any
potential being.\footnote{Adams, ``Must God Create the Best? 319-20.}
Furthermore, beings in the actual but not best world have no right to
complain, lest they express an unmerited claim for special treatment or
violate modest optimism.\footnote{Adams, ``Must God Create the Best? 319-20.} God's omnibenevolence,
therefore, does not demand that he create the best world possible.

As an implication, neutral optimists can entertain that there is a
possible world without natural selection where we exist. However, two
considerations may constrain this possibility. Firstly, this possible
world must be logically coherent. Thomas Morris argues that if God's
omnipotence is committed to what is logically and semantically possible,
God becomes a ``delimiter of possibilities.''\footnote{Thomas V. Morris,
  ``The Necessity of God's Goodness,'' \emph{New Scholasticism} 59
  (1985): 425.} That is, as God's existence is necessary in all possible
worlds, those worlds must reflect his omnipotence by being logically
coherent and his omnibenevolence by being overall good. This means that
if a world without natural selection either fails to be logically
coherent or cannot sustain overall goodness without introducing other
systemic evils, it may not be a genuine possibility after all. Secondly,
this limitation implies that a possible world without natural selection
where we exist is not necessarily better or worse than the actual world.
It could very well be that following the ``only-way'' theodicies, the
goodness of true biological selves must necessarily come through natural
selection and that this outweighs the evil of natural selection.
Regardless, the neutral optimist is distinct in that they can be
grateful for their existence without necessarily implying that natural
selection is instrumentally good.

One obvious challenge against neutral optimism is its shifting
definition of God's omnibenevolence may not be intuitively satisfying.
For instance, Adams's definition of God's ``grace'', which does not
require universal benevolence to all creatures, may only be satisfactory
to some Calvinists or those within certain theological traditions. While
this conception asserts that natural selection does not need to be
justified as instrumentally good, the reality and impact of systemic
evil make it difficult for suffering beings to reconcile that God's
omnibenevolence does not require him to show grace to them, in tension
with their own intuitions about what it means to be loving. However, as
this critique may hold less weight for those aligned with certain
Calvinist doctrines, where such a conception of grace is more readily
accepted, it will be set aside as a doctrinal matter.

A more universal challenge is that even if a neutral optimist can
maintain modest optimism about their existence while affirming systemic
evil through yearning for another possible world, logical constraints on
such worlds mean that regretting the evils of the actual world may
require relinquishing goods unique to its constitution. For example,
recalling Swinburne's free-will theodicy, a possible world without
natural selection might lead to it not having human agency. Similarly,
recalling Southgate's ``only-way'' theodicy, a world without natural
selection could lack independent selves. If the existence of goods like
human agency or autonomous selves carry significant moral weight, then
removing the conditions that produce them (i.e., natural selection) may
render the alternative world no longer overall good---and thus not
genuinely possible. At best, such possible worlds without natural
selection might not involve a loss of goods significant enough to
undermine modest optimism. At worst, the trade-offs could introduce
greater problems of evil. A creationist world, for instance, implies
that God played a direct role in designing cruel beings like the
Ichneumonidae than if they developed independently through evolution.

Comparing extreme and neutral theistic optimism, both conceptions of
modest optimism requires that the world is overall good. This is because
evidence of systemic evils must be outweighed by some other goodness or
burdened with a theodicy. This, however, is not a requirement for
atheist/non-theist optimism.

\section{Two atheist/non-theist modest optimists}
\subsection{Personal optimism}

The first atheist/non-theist modest optimist approach argues that the
scope of existential gratitude can be limited to the personal level
without axiologically considering the world as an aggregate. While
Dawkins expresses his gratitude for existing despite unfavourable odds,
he regrets that, ``Nature is red in tooth and claw. But I don't want to
live in that kind of a world. I want to change the world in which I live
in such a way that natural selection no longer applies.''\footnote{Frank
  Miele, ``Darwin's Dangerous Disciple: An Interview with Richard
  Dawkins,'' \emph{The Skeptic}, 27 October 2010,
  \url{https://www.skeptic.com/eskeptic/10-10-27/}.} However, we can
resolve Dawkins' apparent disjunct by affirming \emph{personal}
existential optimism directed at one's own existence while rejecting
\emph{cosmic} existential optimism that the world is overall good. This
is not methodologically novel; Asha Lancaster-Thomas observes that even
within individuals' lifetimes, we are grateful for some parts of our
lives, but not parts characterised by pain and suffering such as a
painful chronic illness.\footnote{Asha Lancaster-Thomas, ``Can Heaven
  Justify Horrendous Moral Evils? A Postmortem Autopsy,''
  \emph{Religions} 14, No. 296 (2023): 6.}

An implication of personal, but not cosmic, optimism is that their
existential gratitude does not need to consider the axiology of natural
selection. One could remain axiologically agnostic towards the
instruments of their existence, while valuing the goodness of their
personal existence. Guy Kahane emphasises this distinction by arguing
that even if natural selection is a causally fundamental instrument to
our existence, it is axiologically irrelevant as instrumental value
alone does not add any overall value to the world.\footnote{Guy Kahane,
  ``Optimism without theism? Nagasawa on Atheism, Evolution, and Evil,''
  \emph{Religious Studies} 58 (2022): 706.} Under this conception, one
could even be cosmically pessimistic but still be optimistic about their
personal life as they experience it. Modest optimism is thus
reinterpreted to affirm attitudinal optimism, that we are grateful to
exist in this world; but not axiological optimism, that the world is
overall good.\footnote{Guy Kahane, ``Optimism without theism? Nagasawa on Atheism, Evolution, and Evil,'' \emph{Religious Studies} 58 (2022): 702.}

However, after disregarding pessimism, personal optimism appears
empirically challenged as most personal optimists are often implicitly
also cosmic optimists. Responding to Kahane, Nagasawa grants that
personal optimism does not necessarily entail cosmic optimism. However,
he argues that this reformulation of modest optimism changes the target
of POSE, which defines modest optimism as affirming both attitudinal and
axiological optimism.\footnote{Nagasawa, \emph{The Problem of Evil for
  Atheists,} 184.} For he argues that rational personal optimists who
procreate implicitly believe that the world they are bringing their
child into is overall a good place.\footnote{Nagasawa, \emph{The Problem of Evil for
  Atheists,} 184.} The personal, but
not cosmic, reformulation of modest optimism, therefore, seemingly
misses the original target of POSE and is only applicable to a minority
of anti-natalist pessimists like David Benatar.

Responding to this, Nagasawa's formulation of modest optimism is already
limited to the scope of``the environment in which we exist.'' The
specific environment of individual experiences does not necessarily
include the predation experienced by other preyed beings. Indeed, this
does not preclude the modest optimist from being selfish for bringing a
child into the world. Or disregarding the pains of the world, a
personally optimistic individual can choose to be ignorant of the
world's plights by never contributing to charitable causes to use the
money to instead maximise personal pleasures. It is not evident,
therefore, that most personal optimists must also be cosmic optimists.

\subsection{Comparative optimism}

The second atheist/non-theist modest optimist approach argues that
modest optimism only views the world as \emph{comparatively} good, but
not necessarily \emph{categorically} good. That is, the world must only
be \emph{comparatively} better than non-existence, rather than
positively good. This distinction is significant, as Nagasawa's
comparative argument for theism seems to present the axiology of the
world in binary categorical terms. Theism's appeal to a heavenly bliss
allows for a world with more goodness rather than evil.\footnote{Nagasawa, \emph{The Problem of Evil for Atheists,} 171.} But because atheists/non-theists are not committed to affirming
an omnibenevolent God, Kahane argues that they are not obliged to claim
that their existence is categorically good, or that the world contains
more goodness than evil. Indeed, even under Leibniz's extreme optimism,
the world is not necessarily categorically good, just that it is
comparatively the best of all possible worlds.\footnote{Kahane,
  ``Optimism Without Theism,'' 713.}

An implication of a comparatively better, but not categorically good,
optimism is that natural selection does not have to be categorically
good. Assuming that existence in itself is a good greater than all kinds
of non-existence, an actual world with systemic evil is better than any
unactualised world. So, modest optimism's ``not bad'' is equated to
being comparatively better than non-existence. Opposing theism's appeal
to the supernatural, this essentially lowers the requirement for modest
optimism.

One major challenge is that this comparative-goodness version of modest
optimism closely borders on pessimism, and therefore demands an account
of why existence, despite systemic evils, is fundamentally and overall
better than non-existence. The pessimist Benatar, for instance, argues
that the absence of pain is always good, even if no one benefits,
whereas the absence of pleasure is only bad if someone is deprived by
it. This asymmetry supports his claim that existence, with its
inevitable suffering, may be worse than non-existence, which guarantees
goodness with no badness.\footnote{David Benatar, \emph{Better Never to
  Have Been: The Harm of Coming into Existence} (Oxford University
  Press, 2006), 30.}

Responding to Benatar, the optimist can follow Thaddeus Metz's argument
against Benatar's claim that the absence of pain is good, describing the
absence of pain as \emph{not bad} rather than \emph{good.}\footnote{Thaddeus
  Metz, ``Are Lives Worth Creating?'', \emph{Philosophical Papers} 40,
  No. 2 (2011): 241-45} Otherwise, the atheist/non-theist modest
optimist can simply appeal to the previously-discussed personal, rather
than cosmic, optimism. All modest optimism demands is that according to
myself, it is better for me to exist than for me not to exist. Indeed,
Benetar seems to grant this notion, as he distinguishes a present-tense
``life worth continuing'' and future-tense ``life worth
starting.''\footnote{Benatar, \emph{Better Never to Have Been,} 22-23.}
Personal optimists often experience instances where the goods of
actualised pleasure outweigh the evils of pain, resulting in a net
utility that makes existence preferable to non-existence. So, unless one
is personally pessimistic, there is nothing paradoxical about claiming
one's personal life is better to exist than not exist.

Combining these two approaches, the atheist/non-theist, can commit to a
personal and comparative form of modest optimism that still accounts for
the categorically systemic evil of the cosmos. Unlike theistic extreme
optimism's commitment to the instrumental value of natural selection as
a part of God's providence, personal optimists can simply remain
agnostic about natural systems' axiological value. But while theistic
neutral optimists can adopt a similar approach to the
atheism/non-theism's comparative (not categorical) goodness, they remain
committed to both that possible worlds must overall be good, and that
God's creative ability is bound to logical laws, so that the possible
worlds they yearn for must necessarily contain some other kind of
systemic evil that requires a theodicy . The personal optimist on the
other hand need not make this consideration of the overall goodness of
other possible worlds. So, whilst theism can appeal to the heavenly
bliss, the non-theist can simply bypass POSE without needing to address
it.

\section{Borrowing Theism's Optimism Without its Metaphysics}

But even if atheists/non-theists remain burdened by POSE due to perhaps
their cosmic or even categorical optimism, I propose that they can
``borrow'' the theodicies used by theists to justify their modest
optimism. This reverses Nagasawa's theistic strategy, which claims that
theism's supernaturalist ontology (encompassing both natural and
supernatural realms) subsumes the atheist/non-theist's naturalist
ontology (limited to the natural world), thus allowing theists to
``borrow'' atheist/non-theist responses to POSE.\footnote{Nagasawa,
  \emph{The Problem of Evil for Atheists,} 173.} However, Nagasawa does
not address the fact that supernaturalist ontologies bring additional
axiological presuppositions---namely, that an omnibenevolent God exists
and that his creation must necessarily be overall and categorically
good. Non-theists, by contrast, can adopt the theist's belief that the
world is overall good using the theist's rationalisations, without
committing to these broader metaphysical claims about God. In essence,
atheists/non-theists can justify their optimism in the face of POSE
without having to commit to the theist's wider ontological framework.

Borrowing from extreme theistic optimism, the atheist/non-theist can
still view natural selection as categorically good by appealing to the
same free-will and ``only-way'' theodicies---without relying on
theological assumptions. For instance, they may regard natural selection
as instrumentally necessary for the emergence of goods like human
free-will or biological selves and affirm these outcomes as
categorically valuable in themselves. There is nothing inherently
theological in valuing such features of natural history. While theists
might argue that moral value requires an objective grounding in God, the
atheist can respond in two ways: either by offering a naturalistic
foundation for moral value, or by treating such value judgements---and
the modest optimism they support---as subjective, grounded in personal
or shared human perspectives. On this view, modest optimism need not
depend on the objective truth of its content but rather functions as an
attitudinal stance. Accordingly, theist theodicies can be borrowed by
non-theists as explanatory tools, enabling them to affirm the world's
overall goodness without committing to metaphysical claims that theists
traditionally used to justify them.

Borrowing from neutral theistic optimism, the atheist/non-theist can
still affirm that the actual world is not necessarily the best possible
world, but still trust that it is better to exist than not to exist. The
lack of a requirement for atheists/non-theists to commit to the idea
that the world is categorically good allows for a more flexible
position. Even if systemic evils suggest that the world is not
fundamentally good, the personal optimist can still maintain a stance of
cosmic neutrality. They can accept the world as it is---flawed, but not
necessarily bad in a way that undermines their gratitude for existing.
Indeed, without a commitment to an omnibenevolent God who governs over
all creation's actions, the non-theist can simply adopt a position of
gratitude for the outcomes of those processes without ascribing moral or
intrinsic value to these violent/harsh (but not immoral) systemic
processes themselves.

This strategic borrowing highlights a key asymmetry: while theists must
reconcile systemic evil with a metaphysical commitment to a
categorically good creation, non-theists can adopt similar explanatory
frameworks without such constraints. In doing so, they preserve the
practical benefits of modest optimism without incurring the theological
debts that weigh down the theistic response to POSE.

\section*{Conclusion}

POSE, therefore, remains a problem only for theists as their conception
of modest theism must commit to the belief that a good God would create
a categorically good world. This commitment imposes significant burdens
ontheist extreme optimists, whose belief that the actual world is the
best possible world obliges them either to embrace pessimism, appeal to
mystery, or present a theodicy for systemic evils. And while responses
like the free-will and ``only-way'' theodicies may present \emph{prima
facie} defences to POSE, they only regress into deeper manifestations of
the problem of evil unless the theist begs the question or makes an
appeal to mystery. Likewise, theist neutral optimists, who holds that
the actual world is only one of many possible worlds that are not
necessarily the best ones, remain committed to asserting that world is
overall good---which is still difficult to reconcile with or even
amplifies the existence of systemic evils.

In contrast, the atheist/non-theist can either borrow the
theist's theodicies, or maintain a personal comparative optimistic
stance that disregards POSE overall. By selfishly narrowing modest
optimism to the personal level, the atheist/non-theist can disregard
systemic evils while remaining grateful for their own lives as they
experience it. Furthermore, their non-commitment to categorical goodness
allows them to value comparatively their personal lives as better than
non-existence, even if by borrowing neutral optimism, they accept the
world as it is and appreciate the outcomes of systemic processes like
natural selection without assigning moral or intrinsic value to them.

\refsection

\begin{hangparas}{\hangingindent}{1}
Adams, Robert Merrihew. ``Must God Create the Best?''
\emph{Philosophical Review} 81 (1972): 317-332.

Bayle, Pierre. \emph{Historical and Critical Dictionary: Selections.}
Translated by Richard H. Popkin and Craig Brush. Indianapolis, IN:
Hackett, 1991.

Benatar, David. \emph{Better Never to Have Been: The Harm of Coming into
Existence}. Oxford University Press, 2006.

Darwin, Charles. ``22 May 1860 Letter to Asa Gray.'' Darwin
Correspondence Project. Accessed on 5 December 2024.
\newline
\url{https://www.darwinproject.ac.uk/letter/DCP-LETT-2814.xml}.

Dawkins, Richard. ``The Greatest Show on Earth.'' Lecture, University of
Auckland, Auckland, New Zealand, 13 March 2010. Accessed on 3 May 2025.
\newline
\url{https://www.auckland.ac.nz/en/alumni/whats-happening/alumni-video-and-audio/alumni-videos-richard-dawkins.html}.

Rolston III, Holmes. \emph{Science and Religion: A Critical Survey.}
London: Templeton Foundation Press, 2006.

Farrer, Austin. \emph{Love Almighty and Ills Unlimited.} Collins, 1962.

Guy, Kahane. ``Optimism without theism? Nagasawa on atheism, evolution,
and evil.'' \emph{Religious Studies} 58 (2022): 701-714.

Lancaster-Thomas, Asha. ``Can Heaven Justify Horrendous Moral Evils? A
Postmortem Autopsy.'' \emph{Religions} 14, No. 296 (2023).

Leibniz, G. W. \emph{Theodicy.} Edited by Austin Farrer. Translated by
E. M. Huggard. Open Court Publishing Company, 1985.

McFarland, Ian A. \emph{From Nothing: A Theology of Creation.}
Westminster John Knox Press, 2014.

Metz, Thaddeus. ``Are Lives Worth Creating?'' Philosophical Papers 40,
No. 2 (2011): 233-255.

Miele, Frank. ``Darwin's Dangerous Disciple: An Interview with Richard
Dawkins,'' The Skeptic. 27 October 2010.
\newline
\url{https://www.skeptic.com/eskeptic/10-10-27/}.

Morris, Thomas V. ``The Necessity of God's Goodness.'' \emph{New
Scholasticism} 59 (1985): 418-448.

Nagasawa, Yujin. \emph{The Problem of Evil for Atheists.} Oxford
University Press, 2024.

Prescott, Paul. ``The Secular Problem of Evil: An Essay in Analytic
Existentialism.'' \emph{Religious Studies} 57 (2021): 101-119.

Russell, Robert John. ``Natural Theodicy in an Evolutionary Context.''
In \emph{Cosmology: From Alpha to Omega}. Fortress Press, 2008.

Southgate, Christopher. \emph{The Groaning of Creation: God, Evolution,
and the Problem of Evil.} Westminster John Knox Press, 2008.

Swinburne, Richard. ``Natural Evil.'' \emph{American Philosophical
Quarterly} 15\emph{,} No. 4 (1978): 295-301.

Thompson, Janna. ``The Apology Paradox.'' \emph{The Philosophical
Quarterly} 50, No. 201 (2000): 470-475.

Van Inwagen, Peter. ``The Problem of Evil, The Problem of Air, and the
Problem of Silence,'' \emph{Philosophical Perspectives} 5 (1991):
135-165.

\end{hangparas}
