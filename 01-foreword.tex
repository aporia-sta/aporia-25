\vspace*{\credgap}
\begin{center}
    \LARGE \sc Foreword
\end{center}

\vspace{\ackgap}\noindent
Congratulations to all those involved with \emph{Aporia} for reaching
the milestone of a twenty-fifth volume! I was a first-year undergraduate
when the first issue was released. I look back at that day fondly. We
celebrated with a wine reception in Edgecliffe (though, to be entirely
honest, we didn't need much of an occasion to indulge in some wine!),
and there were speeches from Vera Schoeller (the society president) and
Professor Sarah Broadie, who was always a vocal supporter of PhilSoc.

The journal has also published some fantastic work over the years,
including by authors who have gone on to be very successful professional
philosophers. The very first issue included pieces by Marcus Rossberg
and Philip Ebert (now professors at Connecticut and Stirling,
respectively) and a special contribution from Duncan Pritchard (now UC
Irvine). Later issues included Andreas Stokke (Uppsala), Fenner Tanswell
(Technical University Berlin), Steffen Koch (Bielefeld) and Michael
Hicks who is currently my colleague at the University of Glasgow. I was
a co-editor (with Kyle Mitchell) for volumes II and III. I was also
tasked with designing and formatting (which, unfortunately, was not my
forte). Since my time at the helm, I am pleased see that the high
quality of philosophy has continued, and the design quality has
significantly improved. The journal is now an extremely
professional-looking production, and something for all involved to be
proud of.

The St Andrews Philosophy Society will always hold a very dear place in
my heart. It was there that I got to know many people who are my friends
until this day. It is wonderful that the society is still going strong,
and that the journal continues to be a roaring success.

\vspace{\ackgap}\noindent
Dr Joe Slater

\noindent
Lecturer in Moral and Political Philosophy, University of Glasgow

\noindent
St Andrews Philosophy Society President (2011/12)
